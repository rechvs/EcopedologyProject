%Time-stamp: <2017-01-30 18:31:28 renke>
\documentclass[xcolor=dvipsnames]{beamer}

% Einbinden von Paketen: %%%%%%%%%%%%%%%%%%%%%%%%%%%%%%%%%%%%%%%%%%%%%%%%%%%%%%%%%%%%%%%%%%%%%%%%%%%%%%%%%%%%%%%%%%%%%%%%%%%%%%%%%%%%%%%%%%%%%%%%%%%%%%%%%%%%

\usepackage{beamerthemesplit} % Zweiteilung von Kopf- und Fußzeile
%\usepackage[utf8]{inputenc} % Übersetzung von Sonderzeichen usw. in von LaTeX verarbeitbare Befehle
\usepackage[ngerman]{babel} % Typografische (und andere) Regeln sowie Silbentrennung für verschiedene Sprachen
\usepackage{roboto} % Festlegen des Fonts
\usepackage{fixltx2e} % Diverse Verbesserungen von LaTeX
\usepackage{graphicx} % Ermöglicht das Einbinden von Grafiken (bequemer als bei reinem graphics)
\usepackage{url} % Ermöglicht das Einbinden von URLs
\usepackage{xcolor} % Ermöglicht das Erstellen eigener Farben
%\usepackage{textcomp} % LaTeX-Unterstützung für die Text-Companion-Fonts
%\usepackage{gensymb} % Stellt die Befehle \degree, \celsius, \perthousand, \micro und \ohm zur Verfügung, die sowohl in text als auch im math mode funktionieren
\usepackage{calc} % Ermöglicht das Berechnen von Textlängen
\usepackage{tikz} % Ermöglicht das Erstellen von Grafiken in LaTeX
\usepackage{fp} % Ermöglicht einfache arithmetische Berechnungen innerhalb von LaTeX
%\usepackage{pgfpages} % Ermöglicht die Erstellung von "breiten" Folien in beamer

% Einstellungen (paketabhängig): %%%%%%%%%%%%%%%%%%%%%%%%%%%%%%%%%%%%%%%%%%%%%%%%%%%%%%%%%%%%%%%%%%%%%%%%%%%%%%%%%%%%%%%%%%%%%%%%%%%%%%%%%%%%%%%%%%%%%%%%%%%%

\usecolortheme{crane} % Farbschema der Folien (beamer)
\setbeamertemplate{footline}[frame number] % Anzeigen der Foliennummer in der Fußzeile (beamer)
\setbeamertemplate{itemize subitem}{\textcolor{letters}{\textbullet}} % Festlegen der Stichpunktsymbole in Listen (itemize, 2. Ebene, beamer)
\setbeamersize{text margin left=1mm} % Festlegen des Abstands zwischen rechter Kante der linken Sidebar (falls vorhanden, ansonsten linker Kante des Papiers) und linker Kante des Texts
\setbeamersize{text margin right=1mm} % Festlegen des Abstands zwischen linker Kante der rechten Sidebar (falls vorhanden, ansonsten rechter Kante des Papiers) und rechter Kante des Texts
%\setbeameroption{previous slide on second screen=left} % "breite" Folien erstellen (aktuelle Folie links, folgende Folie rechts)

\urlstyle{same} % Festlegen des Fonts für URLs (url)

\definecolor{letters}{HTML}{04064C} % Definieren von Farben (xcolor)
\definecolor{boxes}{HTML}{FCBB06} % Definieren von Farben (xcolor)

\usetikzlibrary{decorations.pathreplacing} % Ermöglicht das Dekorieren von TikZ-Figuren

% Einstellungen (paketunabhängig): %%%%%%%%%%%%%%%%%%%%%%%%%%%%%%%%%%%%%%%%%%%%%%%%%%%%%%%%%%%%%%%%%%%%%%%%%%%%%%%%%%%%%%%%%%%%%%%%%%%%%%%%%%%%%%%%%%%%%%%%%%

%\DeclareMathSizes{12}{20}{8}{10}

% Eigene Kommandos und Umgebungen: %%%%%%%%%%%%%%%%%%%%%%%%%%%%%%%%%%%%%%%%%%%%%%%%%%%%%%%%%%%%%%%%%%%%%%%%%%%%%%%%%%%%%%%%%%%%%%%%%%%%%%%%%%%%%%%%%%%%%%%%%%

\newcommand{\mycaption}[1]{\tiny\raggedright{#1}}
\newcommand*{\myscalebox}[2][0.6]{\scalebox{#1}{\ensuremath{#2}}}%

% Definition von Längen, Breiten, Höhen: %%%%%%%%%%%%%%%%%%%%%%%%%%%%%%%%%%%%%%%%%%%%%%%%%%%%%%%%%%%%%%%%%%%%%%%%%%%%%%%%%%%%%%%%%%%%%%%%%%%%%%%%%%%%%%%%%%%%

\newlength{\balancedcolumnwidth}
\setlength{\balancedcolumnwidth}{0.49\textwidth}

% Definition von Foliennummern zur Steuerung der Sichtbarkeit von Elementen: %%%%%%%%%%%%%%%%%%%%%%%%%%%%%%%%%%%%%%%%%%%%%%%%%%%%%%%%%%%%%%%%%%%%%%%%%%%%%%%%
\newcounter{firstElement}
\newcounter{secondElement}
\newcounter{thirdElement}
\newcounter{fourthElement}
\newcounter{fifthElement}
\newcounter{sixthElement}
\newcounter{seventhElement}
\newcounter{eighthElement}
\newcounter{ninthElement}
\newcounter{tenthElement}
\FPset{\thefirstElement}{2}
\FPadd{\thesecondElement}{\thefirstElement}{1}
\FPadd{\thethirdElement}{\thesecondElement}{1}
\FPadd{\thefourthElement}{\thethirdElement}{1}
\FPadd{\thefifthElement}{\thefourthElement}{1}
\FPadd{\thesixthElement}{\thefifthElement}{1}
\FPadd{\theseventhElement}{\thesixthElement}{1}
\FPadd{\theeighthElement}{\theseventhElement}{1}
\FPadd{\theninthElement}{\theeighthElement}{1}
\FPadd{\thetenthElement}{\theninthElement}{1}
\FPround{\thesecondElement}{\thesecondElement}{0}
\FPround{\thethirdElement}{\thethirdElement}{0}
\FPround{\thefourthElement}{\thefourthElement}{0}
\FPround{\thefifthElement}{\thefifthElement}{0}
\FPround{\thesixthElement}{\thesixthElement}{0}
\FPround{\theseventhElement}{\theseventhElement}{0}
\FPround{\theeighthElement}{\theeighthElement}{0}
\FPround{\theninthElement}{\theninthElement}{0}
\FPround{\thetenthElement}{\thetenthElement}{0}

% Beginn des Dokuments: %%%%%%%%%%%%%%%%%%%%%%%%%%%%%%%%%%%%%%%%%%%%%%%%%%%%%%%%%%%%%%%%%%%%%%%%%%%%%%%%%%%%%%%%%%%%%%%%%%%%%%%%%%%%%%%%%%%%%%%%%%%%%%%%%%%%%

\begin{document}

\title{Transekte des Wasserhaushalts in Agroforstsystemen}
\subtitle{Vortrag zum Projekt \\ ,,Ökosystemanalyse und Modellierung``}
\author{Renke von Seggern}
\date[23.01.2017]{23. Januar 2017}

\beamertemplatenavigationsymbolsempty

\begin{frame}[plain]
\titlepage
\end{frame}

% \addtocounter{framenumber}{-2} %Reduktion der Folienzahl (-1 für Titelfolie, -1 für Inthaltsübersicht)

% \begin{frame}[plain]
% \frametitle{Inhaltsübersicht}
% \end{frame}

\addtocounter{framenumber}{-1} %Reduktion der Folienzahl (-1 für Titelfolie)

\section{Einleitung}
\subsection{Ziele, Probleme, Lösungsansätze}
\begin{frame}[t]
  \only<1>{
    \centerline{
      \begin{minipage}{0.9\textwidth}
        \includegraphics[width=1.0\textwidth]{Grafiken/IMG_1181_beschnitten} \\
        \mycaption{Abbildung: Beispiel für das Landnutzungsschema der Untersuchungsflächen (hier: KUP und Grünland)
          (Quelle: Falk Richter, verändert).}
      \end{minipage}}}

  \begin{itemize}
  \item<\thefirstElement-> Ziel: \\
    Modellierung des Bodenwasserhaushalts von Ackerkulturen in Nachbarschaft zu KUPs.
  \item<\thesecondElement-> Probleme:
    \begin{itemize}
    \item<\thesecondElement-> Einfluss von KUPs auf benachbarte Flächen \\
      $\boldsymbol\Rightarrow$ keine trennscharfen Grenzen
    \item<\thesecondElement-> nichtlineare Abhängigkeiten, vielfältige Einflussgrößen \\
      $\boldsymbol\Rightarrow$ keine analytische Lösung möglich
    \end{itemize}
  \end{itemize}
  \begin{itemize}
  \item<\thethirdElement-> Lösungsansatz:
    \begin{itemize}
    \item<\thethirdElement-> räumlich kleinskalige Modellierung
    \item<\thethirdElement-> implizite numerische Lösung (finite Differenzen + Picard-Iteration)
    \item<\thethirdElement-> Software: GNU Octave
    \end{itemize}
  \end{itemize}
\end{frame}

\subsection{Unterschiede gegenüber älteren Modellen}
\begin{frame}[t]
  \only<1>{
    \centerline{
      \begin{minipage}{0.9\textwidth}  
        \includegraphics[width=1.0\textwidth]{Grafiken/alter_ansatz} \\
        \mycaption{Abbildung: Schematische Darstellung bisheriger Modellansätze, die von trennscharfen Grenzen ausgehen
          (Quelle: eigenes Werk).}
      \end{minipage}}}

  \only<2>{
    \centerline{
      \begin{minipage}{0.9\textwidth}  
        \includegraphics[width=1.0\textwidth]{Grafiken/neuer_ansatz01} \\
        \mycaption{Abbildung: Schematische Darstellung des zu entwickelnden Modells
          (Quelle: eigenes Werk).}
      \end{minipage}}}
  
  \only<3>{
    \centerline{
      \begin{minipage}{0.9\textwidth}  
        \includegraphics[width=1.0\textwidth]{Grafiken/neuer_ansatz02} \\
        \mycaption{Abbildung: Schematische Darstellung des zu entwickelnden Modells
          (Quelle: eigenes Werk).}
      \end{minipage}}}
\end{frame}

\section{Modellschema}
\begin{frame}[c]
  \centerline{
    % \fbox{
    \begin{tikzpicture}[>=stealth]
      % tn
      \visible<\thesecondElement->{\draw (0,0) node{\tiny$t_n$};}
      % Pfeil, Delta t
      \visible<\thethirdElement->{\draw[->] (0.5,0) -- (3.5,0);}
      \visible<\thethirdElement->{\draw (2,0.5) node{\tiny$\Delta t$};}
      % tn+1
      \visible<\thethirdElement->{\draw (4,0) node{\tiny$t_{n+1}$};}
      % m=1
      \visible<\thefourthElement->{\draw (1,-0.5) node{\tiny$m = 1$};}
      % m=2
      \visible<\thefourthElement->{\draw (2,-0.5) node{\tiny$m = 2$};}
      % Auslassungspunkte für m=3
      \visible<\thefourthElement->{\draw (3,-0.5) node{\tiny\ldots};}
      % % senkrechte Linien (tn)
      \visible<\thefirstElement->{\draw[gray,thin] (0,-1) -- (0,-2);}
      \visible<\thefirstElement->{\draw[gray,thin] (0,-3) -- (0,-5);}
      % % Auslassungspunkte (tn)
      \visible<\thefirstElement->{\draw (0,-2.4) node{\vdots};}
      % Knoten (tn)
      \foreach \ycoord in {-1,-2}
      {
        \visible<\thefirstElement->{\fill[black] (0,\ycoord) circle (0.75mm);}
      }
      \foreach \ycoord in {-3,-4,-5}
      {
        \visible<\thefirstElement->{\fill[black] (0,\ycoord) circle (0.75mm);}
      }
      % % i=
      \visible<\thesecondElement->{\draw (-1,-1) node{\tiny$i =$};}
      % Knotennummern
      \visible<\thesecondElement->{\draw (-0.5,-1) node{\tiny$x$};}
      \visible<\thesecondElement->{\draw (-0.5,-2) node{\tiny$x-1$};}
      \visible<\thesecondElement->{\draw (-0.5,-3) node{\tiny$2$};}
      \visible<\thesecondElement->{\draw (-0.5,-4) node{\tiny$1$};}
      \visible<\thesecondElement->{\draw (-0.5,-5) node{\tiny$0$};}
      % % Delta z
      \visible<\thesecondElement->{\draw (0.35,-4.5) node{\tiny$\Delta z$};}
      % geschwungene Klammer (zu Delta z)
      \visible<\thesecondElement->{\draw[decorate,decoration=brace] (0.075,-4) -- (0.075,-5);}
      % % senkrechte Linien (m=1)
      \visible<\thefourthElement->{\draw[gray,thin] (1,-1) -- (1,-2);}
      \visible<\thefourthElement->{\draw[gray,thin] (1,-3) -- (1,-5);}
      % % Auslassungspunkte (m=1)
      \visible<\thefourthElement->{\draw[gray] (1,-2.4) node{\vdots};}
      % Knoten (m=1)
      \foreach \ycoord in {-1,-2}
      {
        \visible<\thefourthElement->{\fill[gray] (1,\ycoord) circle (0.5mm);}
      }
      \foreach \ycoord in {-3,-4,-5}
      {
        \visible<\thefourthElement->{\fill[gray] (1,\ycoord) circle (0.5mm);}
      }
      % % senkrechte Linien (m=2)
      \visible<\thefourthElement->{\draw[gray,thin] (2,-1) -- (2,-2);}
      \visible<\thefourthElement->{\draw[gray,thin] (2,-3) -- (2,-5);}
      % % Auslassungspunkte (m=2)
      \visible<\thefourthElement->{\draw[gray] (2,-2.4) node{\vdots};}
      % Knoten (m=2)
      \foreach \ycoord in {-1,-2}
      {
        \visible<\thefourthElement->{\fill[gray] (2,\ycoord) circle (0.5mm);}
      }
      \foreach \ycoord in {-3,-4,-5}
      {
        \visible<\thefourthElement->{\fill[gray] (2,\ycoord) circle (0.5mm);}
      }
      % % senkrechte Linien (tn+1)
      \visible<\thethirdElement->{\draw[gray,thin] (4,-1) -- (4,-2);}
      \visible<\thethirdElement->{\draw[gray,thin] (4,-3) -- (4,-5);}
      % % Auslassungspunkte (tn+1)
      \visible<\thethirdElement->{\draw (4,-2.4) node{\vdots};}
      % Knoten (tn+1)
      \foreach \ycoord in {-1,-2}
      {
        \visible<\thethirdElement->{\fill[black] (4,\ycoord) circle (0.75mm);}
      }
      \foreach \ycoord in {-3,-4,-5}
      {
        \visible<\thethirdElement->{\fill[black] (4,\ycoord) circle (0.75mm);}
      }
    \end{tikzpicture}}
  % }
  \visible<\thefirstElement->{
    \centerline{
      \parbox{0.7\textwidth}{
        \mycaption{Abbildung: Schematische Darstellung des zugrundeliegenden Knotenmodells und des Modellablaufs.}}}}
\end{frame}

\section{Gleichungssystem}
\subsection{Richards-Gleichung $\boldsymbol{\rightarrow}$ Näherung $\boldsymbol{\rightarrow}$ Matrixschreibweise}

\begin{frame}
  \visible<\thefirstElement->{\frametitle{Richards-Gleichung}}
  \only<\thefirstElement>{
    \centerline{
      % $C(h) \frac{\delta h}{\delta t} - \nabla \cdot K(h)\nabla h - \frac{\delta K}{\delta z} = 0$
      $\frac{\delta \theta}{\delta t} - \nabla \cdot K(h)\nabla h - \frac{\delta K}{\delta z} = 0$
      \vspace{5mm}}
    \begin{minipage}{1.0\textwidth}
    \centerline{
      % \mycaption{Gleichung: h-basierte Richards-Gleichung (Quelle: [1]).}}
      \mycaption{Gleichung: Gemischte Richards-Gleichung (Quelle: [1]).}}
    \end{minipage}}

  % \only<\thethirdElement->{
  %   \centerline{
  %     $C(h) \textcolor{red}{\frac{\delta h}{\delta t}} - \nabla \cdot K(h)\nabla h -
  %     \frac{\delta K}{\delta z} = 0$
  %   \vspace{5mm}}}
  \only<\thefirstElement->{
    \begin{align*}
      % \myscalebox{C(h)} & \myscalebox{: \text{spezifische Feuchtekapazitätsfunktion [m\textsuperscript{-1}]}} \\[-3mm]
      \myscalebox{h} & \myscalebox{: \text{Wasserspannung [m]}} \\[-3mm]
      \myscalebox{K(h)} & \myscalebox{: \text{ungesättigte hydraulische Leitfähigkeit [m $\cdot$ s\textsuperscript{-1}]}} \\[-3mm]
      \myscalebox{t} & \myscalebox{: \text{Zeit [s]}} \\[-3mm]
      \myscalebox{\theta} & \myscalebox{: \text{Wassergehalt [m\textsuperscript{3} $\cdot$ m\textsuperscript{-3}]}} \\[-3mm]
      \myscalebox{z} & \myscalebox{: \text{vertikale Distanz [m]}} \\[-3mm]
    \end{align*}}
\end{frame}

\begin{frame}
  \frametitle{Näherung (finite Differenzen + Picard-Iteration)}
  \only<\thefirstElement>{
    \vspace{-5mm}
    \begin{gather*}
      % C_i^{n+1,m} \cdot \frac{\delta_i^m}{\Delta t} - \frac{1}{\left(\Delta z\right)^2} \cdot \left[K_{i+\frac{1}{2}}^{n+1,m} \cdot \left(\delta_{i+1}^m - \delta_i^m\right) - K_{i-\frac{1}{2}}^{n+1,m} \cdot \left(\delta_i^m - \delta_{i-1}^m\right)\right] \\[+0.5mm]
      % = \\[+0.5mm]
      % \frac{1}{\left(\Delta z\right)^2} \cdot \left[K_{i+\frac{1}{2}}^{n+1,m} \cdot \left(H_{i+1}^{n+1,m} - H_i^{n+1,m}\right) - K_{i-\frac{1}{2}}^{n+1,m} \cdot \left(H_{i}^{n+1,m} - H_{i-1}^{n+1,m}\right)\right] \\[+0.5mm]
      % + \frac{K_{i+\frac{1}{2}}^{n+1,m} - K_{i-\frac{1}{2}}^{n+1,m}}{\Delta z} - C_i^{n+1,m} \cdot \frac{H_i^{n+1,m} - H_i^n}{\Delta t}
      C_i^{n+1,m} \cdot \frac{\delta_i^m}{\Delta t} - \frac{1}{\left(\Delta z\right)^2} \cdot \left[K_{i+\frac{1}{2}}^{n+1,m} \cdot \left(\delta_{i+1}^m - \delta_i^m\right) - K_{i-\frac{1}{2}}^{n+1,m} \cdot \left(\delta_i^m - \delta_{i-1}^m\right)\right] \\[+0.5mm]
      = \\[+0.5mm]
      \frac{1}{\left(\Delta z\right)^2} \cdot \left[K_{i+\frac{1}{2}}^{n+1,m} \cdot \left(H_{i+1}^{n+1,m} - H_i^{n+1,m}\right) - K_{i-\frac{1}{2}}^{n+1,m} \cdot \left(H_{i}^{n+1,m} - H_{i-1}^{n+1,m}\right)\right] \\[+0.5mm]
      + \frac{K_{i+\frac{1}{2}}^{n+1,m} - K_{i-\frac{1}{2}}^{n+1,m}}{\Delta z} - \frac{\theta_i^{n+1,m}-\theta_i^{n}}{\Delta t}
    \end{gather*}
      \begin{minipage}{1.0\textwidth}
        \mycaption{Gleichung: Standardmäßige (räumliche) Näherung zur Richards-Gleichung mittels finiter Differenzen nach Anwendung der Picard-Iteration (Quelle: [1]).}
      \end{minipage}
    \vspace{-7mm}}
  \only<\thesecondElement->{
    \vspace{-5mm}
    \begin{gather*}
      % C_i^{n+1,m} \cdot \frac{\textcolor{red}{\delta_i^m}}{\Delta t} - \frac{1}{\left(\Delta z\right)^2} \cdot \left[K_{i+\frac{1}{2}}^{n+1,m} \cdot \left(\textcolor{red}{\delta_{i+1}^m} - \textcolor{red}{\delta_i^m}\right) - K_{i-\frac{1}{2}}^{n+1,m} \cdot \left(\textcolor{red}{\delta_i^m} - \textcolor{red}{\delta_{i-1}^m}\right)\right] \\[+0.5mm]
      % = \\[+0.5mm]
      % \frac{1}{\left(\Delta z\right)^2} \cdot \left[K_{i+\frac{1}{2}}^{n+1,m} \cdot \left(H_{i+1}^{n+1,m} - H_i^{n+1,m}\right) - K_{i-\frac{1}{2}}^{n+1,m} \cdot \left(H_{i}^{n+1,m} - H_{i-1}^{n+1,m}\right)\right] \\[+0.5mm]
      % + \frac{K_{i+\frac{1}{2}}^{n+1,m} - K_{i-\frac{1}{2}}^{n+1,m}}{\Delta z} - C_i^{n+1,m} \cdot \frac{H_i^{n+1,m} - H_i^n}{\Delta t}
      C_i^{n+1,m} \cdot \frac{\textcolor{red}{\delta_i^m}}{\Delta t} - \frac{1}{\left(\Delta z\right)^2} \cdot \left[K_{i+\frac{1}{2}}^{n+1,m} \cdot \left(\textcolor{red}{\delta_{i+1}^m} - \textcolor{red}{\delta_i^m}\right) - K_{i-\frac{1}{2}}^{n+1,m} \cdot \left(\textcolor{red}{\delta_i^m} - \textcolor{red}{\delta_{i-1}^m}\right)\right] \\[+0.5mm]
      = \\[+0.5mm]
      \frac{1}{\left(\Delta z\right)^2} \cdot \left[K_{i+\frac{1}{2}}^{n+1,m} \cdot \left(H_{i+1}^{n+1,m} - H_i^{n+1,m}\right) - K_{i-\frac{1}{2}}^{n+1,m} \cdot \left(H_{i}^{n+1,m} - H_{i-1}^{n+1,m}\right)\right] \\[+0.5mm]
      + \frac{K_{i+\frac{1}{2}}^{n+1,m} - K_{i-\frac{1}{2}}^{n+1,m}}{\Delta z} - \frac{\theta_i^{n+1,m}-\theta_i^{n}}{\Delta t}
    \end{gather*}
      \begin{minipage}{1.0\textwidth}
        \mycaption{Gleichung: Standardmäßige (räumliche) Näherung zur Richards-Gleichung mittels finiter Differenzen nach Anwendung der Picard-Iteration (Quelle: [1]).}
      \end{minipage}
    \vspace{-7mm}}
  \visible<\thefirstElement->{
    \begin{columns}
      \begin{column}{0.5\textwidth}
        \begin{align*}
          \myscalebox{C} & \myscalebox{: \text{spezifische Feuchtekapazität [m\textsuperscript{-1}]}} \\[-3mm]
          \only<\thefirstElement>{\myscalebox{\delta_i^m} & \myscalebox{: H_i^{n+1,m+1} - H_i^{n+1,m}} \\[-3mm]}
          \only<\thesecondElement->{\myscalebox{\textcolor{red}{\delta_i^m}} & \myscalebox{\textcolor{red}{: H_i^{n+1,m+1} - H_i^{n+1,m}}} \\[-3mm]}
          \myscalebox{H} & \myscalebox{: \text{(genäherte) Wasserspannung [m]}} \\[-3mm]
          \myscalebox{i} & \myscalebox{: \text{Knotennummer} (0 \ldots x)} \\[-3mm]
          \myscalebox{K} & \myscalebox{: \text{hydraulische Leitfähigkeit [m $\cdot$ s\textsuperscript{-1}]}} \\[-3mm]
          % \myscalebox{m} & \myscalebox{: \text{Iterationsschritt}} \\[-3mm]
          % \myscalebox{n} & \myscalebox{: \text{Zeitschritt}} \\[-3mm]
          % \myscalebox{t} & \myscalebox{: \text{Zeit [s]}} \\[-3mm]
          % \myscalebox{z} & \myscalebox{: \text{Strecke [m]}} \\[-3mm]
        \end{align*}
    \end{column}
      \begin{column}{0.5\textwidth}
        \begin{align*}
          % \myscalebox{C} & \myscalebox{: \text{spezifische Feuchtekapazität [m\textsuperscript{-1}]}} \\[-3mm]
          % \only<\thefirstElement>{\myscalebox{\delta_i^m} & \myscalebox{: H_i^{n+1,m+1} - H_i^{n+1,m}} \\[-3mm]}
          % \only<\thesecondElement->{\myscalebox{\textcolor{red}{\delta_i^m}} & \myscalebox{\textcolor{red}{: H_i^{n+1,m+1} - H_i^{n+1,m}}} \\[-3mm]}
          % \myscalebox{H} & \myscalebox{: \text{(genäherte) Wasserspannung [m $\cdot$ s\textsuperscript{-1}]}} \\[-3mm]
          % \myscalebox{i} & \myscalebox{: \text{Knotennummer} (0 \ldots x)} \\[-3mm]
          % \myscalebox{K} & \myscalebox{: \text{hydraulische Leitfähigkeit [m $\cdot$ s\textsuperscript{-1}]}} \\[-3mm]
          \myscalebox{m} & \myscalebox{: \text{Iterationsschritt}} \\[-3mm]
          \myscalebox{n} & \myscalebox{: \text{Zeitschritt}} \\[-3mm]
          \myscalebox{t} & \myscalebox{: \text{Zeit [s]}} \\[-3mm]
          \myscalebox{\theta} & \myscalebox{: \text{Wassergehalt [m\textsuperscript{3} $\cdot$ m\textsuperscript{-3}]}} \\[-3mm]
          \myscalebox{z} & \myscalebox{: \text{Strecke [m]}} \\[-3mm]
        \end{align*}}
    \end{column}
  \end{columns}

\end{frame}

\begin{frame}[t]
  \frametitle{Matrixschreibweise}
  \only<\thefirstElement-\thesecondElement>{
    \centerline{
      $A \cdot \delta = f$}}
  \only<\thethirdElement>{
    \centerline{
      $A \cdot \textcolor{red}{\delta} = f$}}

  \only<\thefirstElement>{
    \vspace{2mm}
    \centerline{
      \begin{math}
        A =
        \begin{pmatrix}
          \textcolor{gray}{a_{11}} & \textcolor{gray}{a_{12}} & \textcolor{gray}{a_{13}} \\
          %\only<\thesecondElement>{a_{21} & a_{22} & a_{23} \\}
          \textcolor{black}{a_{21}} & \textcolor{black}{a_{22}} & \textcolor{black}{a_{23}} \\
          \textcolor{gray}{a_{31}} & \textcolor{gray}{a_{32}} & \textcolor{gray}{a_{33}} \\
        \end{pmatrix}
        ;
        \delta =
        \begin{pmatrix}
          \textcolor{black}{\delta_1} \\
          \textcolor{black}{\delta_2} \\
          \textcolor{black}{\delta_3} \\
        \end{pmatrix}
        ;
        f =
        \begin{pmatrix}
          \textcolor{gray}{f_1} \\
          \textcolor{black}{f_2} \\
          \textcolor{gray}{f_3} \\
        \end{pmatrix}
      \end{math}}
    \begin{align*}
      % f_1 & = a_{11} \cdot \delta_1 + a_{12} \cdot \delta_2 + a_{13} \cdot \delta_3 \\
      %\only<\thesecondElement>{f_2 & = a_{21} \cdot \delta_1 + a_{22} \cdot \delta_2 + a_{23} \cdot \delta_3 \\}
      \textcolor{black}{f_2} & = \textcolor{black}{a_{21}} \cdot \textcolor{black}{\delta_1} + \textcolor{black}{a_{22}} \cdot \textcolor{black}{\delta_2} + \textcolor{black}{a_{23}} \cdot \textcolor{black}{\delta_3} \\
      % f_3 & = a_{31} \cdot \delta_3 + a_{32} \cdot \delta_3 + a_{33} \cdot \delta_3 \\
    \end{align*}}

  \only<\thesecondElement>{
    \begin{gather*}
      % C_i^{n+1,m} \cdot \frac{\delta_i^m}{\Delta t} - \frac{1}{\left(\Delta z\right)^2} \cdot \left[K_{i+\frac{1}{2}}^{n+1,m} \cdot \left(\delta_{i+1}^m - \delta_i^m\right) - K_{i-\frac{1}{2}}^{n+1,m} \cdot \left(\delta_i^m - \delta_{i-1}^m\right)\right] \\[+0.5mm]
      % = \\[+0.5mm]
      % \frac{1}{\left(\Delta z\right)^2} \cdot \left[K_{i+\frac{1}{2}}^{n+1,m} \cdot \left(H_{i+1}^{n+1,m} - H_i^{n+1,m}\right) - K_{i-\frac{1}{2}}^{n+1,m} \cdot \left(H_{i}^{n+1,m} - H_{i-1}^{n+1,m}\right)\right] \\[+0.5mm]
      % + \frac{K_{i+\frac{1}{2}}^{n+1,m} - K_{i-\frac{1}{2}}^{n+1,m}}{\Delta z} - C_i^{n+1,m} \cdot \frac{H_i^{n+1,m} - H_i^n}{\Delta t}
      C_i^{n+1,m} \cdot \frac{\delta_i^m}{\Delta t} - \frac{1}{\left(\Delta z\right)^2} \cdot \left[K_{i+\frac{1}{2}}^{n+1,m} \cdot \left(\delta_{i+1}^m - \delta_i^m\right) - K_{i-\frac{1}{2}}^{n+1,m} \cdot \left(\delta_i^m - \delta_{i-1}^m\right)\right] \\[+0.5mm]
      = \\[+0.5mm]
      \frac{1}{\left(\Delta z\right)^2} \cdot \left[K_{i+\frac{1}{2}}^{n+1,m} \cdot \left(H_{i+1}^{n+1,m} - H_i^{n+1,m}\right) - K_{i-\frac{1}{2}}^{n+1,m} \cdot \left(H_{i}^{n+1,m} - H_{i-1}^{n+1,m}\right)\right] \\[+0.5mm]
      + \frac{K_{i+\frac{1}{2}}^{n+1,m} - K_{i-\frac{1}{2}}^{n+1,m}}{\Delta z} - \frac{\theta_i^{n+1,m}-\theta_i^{n}}{\Delta t}
    % \end{gather*}
    % \begin{gather*}
    %   C_i^{n+1,m} \cdot \frac{\textcolor{red}{\delta_i^m}}{\Delta t} - \frac{1}{\left(\Delta z\right)^2} \cdot \left[K_{i+\frac{1}{2}}^{n+1,m} \cdot \left(\textcolor{red}{\delta_{i+1}^m} - \textcolor{red}{\delta_i^m}\right) - K_{i-\frac{1}{2}}^{n+1,m} \cdot \left(\textcolor{red}{\delta_i^m} - \textcolor{red}{\delta_{i-1}^m}\right)\right] \\[+0.5mm]
    %   = \\[+0.5mm]
    %   \frac{1}{\left(\Delta z\right)^2} \cdot \left[K_{i+\frac{1}{2}}^{n+1,m} \cdot \left(H_{i+1}^{n+1,m} - H_i^{n+1,m}\right) - K_{i-\frac{1}{2}}^{n+1,m} \cdot \left(H_{i}^{n+1,m} - H_{i-1}^{n+1,m}\right)\right] \\[+0.5mm]
    %   + \frac{K_{i+\frac{1}{2}}^{n+1,m} -
    %     K_{i-\frac{1}{2}}^{n+1,m}}{\Delta z} - C_i^{n+1,m} \cdot
    %   \frac{H_i^{n+1,m} - H_i^n}{\Delta t}
    \end{gather*}}

\only<\thethirdElement>{
    \begin{gather*}
      % C_i^{n+1,m} \cdot \frac{\textcolor{red}{\delta_i^m}}{\Delta t} - \frac{1}{\left(\Delta z\right)^2} \cdot \left[K_{i+\frac{1}{2}}^{n+1,m} \cdot \left(\textcolor{red}{\delta_{i+1}^m} - \textcolor{red}{\delta_i^m}\right) - K_{i-\frac{1}{2}}^{n+1,m} \cdot \left(\textcolor{red}{\delta_i^m} - \textcolor{red}{\delta_{i-1}^m}\right)\right] \\[+0.5mm]
      % = \\[+0.5mm]
      % \frac{1}{\left(\Delta z\right)^2} \cdot \left[K_{i+\frac{1}{2}}^{n+1,m} \cdot \left(H_{i+1}^{n+1,m} - H_i^{n+1,m}\right) - K_{i-\frac{1}{2}}^{n+1,m} \cdot \left(H_{i}^{n+1,m} - H_{i-1}^{n+1,m}\right)\right] \\[+0.5mm]
      % + \frac{K_{i+\frac{1}{2}}^{n+1,m} - K_{i-\frac{1}{2}}^{n+1,m}}{\Delta z} - C_i^{n+1,m} \cdot \frac{H_i^{n+1,m} - H_i^n}{\Delta t}
      C_i^{n+1,m} \cdot \frac{\textcolor{red}{\delta_i^m}}{\Delta t} - \frac{1}{\left(\Delta z\right)^2} \cdot \left[K_{i+\frac{1}{2}}^{n+1,m} \cdot \left(\textcolor{red}{\delta_{i+1}^m} - \textcolor{red}{\delta_i^m}\right) - K_{i-\frac{1}{2}}^{n+1,m} \cdot \left(\textcolor{red}{\delta_i^m} - \textcolor{red}{\delta_{i-1}^m}\right)\right] \\[+0.5mm]
      = \\[+0.5mm]
      \frac{1}{\left(\Delta z\right)^2} \cdot \left[K_{i+\frac{1}{2}}^{n+1,m} \cdot \left(H_{i+1}^{n+1,m} - H_i^{n+1,m}\right) - K_{i-\frac{1}{2}}^{n+1,m} \cdot \left(H_{i}^{n+1,m} - H_{i-1}^{n+1,m}\right)\right] \\[+0.5mm]
      + \frac{K_{i+\frac{1}{2}}^{n+1,m} - K_{i-\frac{1}{2}}^{n+1,m}}{\Delta z} - \frac{\theta_i^{n+1,m}-\theta_i^{n}}{\Delta t}
    \end{gather*}}

\end{frame}

\subsection{Matrix- und Vektorenelemente}  

\begin{frame}[t]
  \frametitle{Matrix $A$}
  % \only<\thefirstElement-\thesecondElement>{
    % \centerline{
      % \begin{math}
        % A =
        % \begin{pmatrix}
          % b_0 & c_0 & & & & \\
          % a_1 & b_1 & c_1 & & & \\
          % & a_2 & b_2 & c_2 & & \\
          % & & \ddots & \ddots & \ddots & \\
          % & & & a_{x-1} & b_{x-1} & c_{x-1} \\
          % & & & & a_x & b_x
        % \end{pmatrix}
      % \end{math}}}
  % \only<\thefirstElement>{
    % \begin{gather*}
      % C_i^{n+1,m} \cdot \frac{\textcolor{red}{\delta_i^m}}{\Delta t} - \frac{1}{\left(\Delta z\right)^2} \cdot \left[K_{i+\frac{1}{2}}^{n+1,m} \cdot \left(\textcolor{red}{\delta_{i+1}^m} - \textcolor{red}{\delta_i^m}\right) - K_{i-\frac{1}{2}}^{n+1,m} \cdot \left(\textcolor{red}{\delta_i^m} - \textcolor{red}{\delta_{i-1}^m}\right)\right] \\[+1mm]
      % = \\[+1mm]
      % \textcolor{black}{C_i^{n+1,m} \cdot \frac{1}{\Delta t}} \cdot \textcolor{red}{\delta_i^m}
      % \textcolor{black}{ - \frac{K_{i+\frac{1}{2}}^{n+1,m}}{\left(\Delta z\right)^2}} \cdot \textcolor{red}{\delta_{i+1}^m}
      % \textcolor{black}{ + \frac{K_{i+\frac{1}{2}}^{n+1,m}}{\left(\Delta z\right)^2}} \cdot \textcolor{red}{\delta_i^m}
      % \textcolor{black}{ + \frac{K_{i-\frac{1}{2}}^{n+1,m}}{\left(\Delta z\right)^2}} \cdot \textcolor{red}{\delta_i^m}
      % \textcolor{black}{ - \frac{K_{i-\frac{1}{2}}^{n+1,m}}{\left(\Delta z\right)^2}} \cdot \textcolor{red}{\delta_{i-1}^m}
    % \end{gather*}}

  \only<\thefirstElement->{
    \centerline{
      \begin{math}
        A =
        \begin{pmatrix}
          \textcolor{green}{b_0} & \textcolor{orange}{c_0} & & & & \\
          \textcolor{blue}{a_1} & \textcolor{green}{b_1} & \textcolor{orange}{c_1} & & & \\
          & \textcolor{blue}{a_2} & \textcolor{green}{b_2} & \textcolor{orange}{c_2} & & \\
          & & \ddots & \ddots & \ddots & \\
          & & & \textcolor{blue}{a_{x-1}} & \textcolor{green}{b_{x-1}} & \textcolor{orange}{c_{x-1}} \\
          & & & & \textcolor{blue}{a_x} & \textcolor{green}{b_x}
        \end{pmatrix}
      \end{math}}}
  \only<\thefirstElement->{
    \begin{gather*}
      C_i^{n+1,m} \cdot \frac{\textcolor{red}{\delta_i^m}}{\Delta t} - \frac{1}{\left(\Delta z\right)^2} \cdot \left[K_{i+\frac{1}{2}}^{n+1,m} \cdot \left(\textcolor{red}{\delta_{i+1}^m} - \textcolor{red}{\delta_i^m}\right) - K_{i-\frac{1}{2}}^{n+1,m} \cdot \left(\textcolor{red}{\delta_i^m} - \textcolor{red}{\delta_{i-1}^m}\right)\right] \\[+1mm]
      = \\[+1mm]
      \textcolor{green}{C_i^{n+1,m} \cdot \frac{1}{\Delta t}} \cdot \textcolor{red}{\delta_i^m}
      \textcolor{orange}{ - \frac{K_{i+\frac{1}{2}}^{n+1,m}}{\left(\Delta z\right)^2}} \cdot \textcolor{red}{\delta_{i+1}^m}
      \textcolor{green}{ + \frac{K_{i+\frac{1}{2}}^{n+1,m}}{\left(\Delta z\right)^2}} \cdot \textcolor{red}{\delta_i^m}
      \textcolor{green}{ + \frac{K_{i-\frac{1}{2}}^{n+1,m}}{\left(\Delta z\right)^2}} \cdot \textcolor{red}{\delta_i^m}
      \textcolor{blue}{ - \frac{K_{i-\frac{1}{2}}^{n+1,m}}{\left(\Delta z\right)^2}} \cdot \textcolor{red}{\delta_{i-1}^m}
    \end{gather*}}
\end{frame}

\begin{frame}[t]
  \frametitle{Vektor $\delta$}
  \visible<1->{
    \centerline{
      \begin{math}
        \delta =
        \begin{pmatrix}
          \delta_0 \\
          \delta_1 \\
          \delta_2 \\
          \vdots \\
          \delta_{x-1} \\
          \delta_x
        \end{pmatrix}
      \end{math}}
    \begin{gather*}
      C_i^{n+1,m} \cdot \frac{\textcolor{red}{\delta_i^m}}{\Delta t} - \frac{1}{\left(\Delta z\right)^2} \cdot \left[K_{i+\frac{1}{2}}^{n+1,m} \cdot \left(\textcolor{red}{\delta_{i+1}^m} - \textcolor{red}{\delta_i^m}\right) - K_{i-\frac{1}{2}}^{n+1,m} \cdot \left(\textcolor{red}{\delta_i^m} - \textcolor{red}{\delta_{i-1}^m}\right)\right]
    \end{gather*}}
\end{frame}

\begin{frame}[t]
  \frametitle{Vektor $f$}
  \visible<1->{
    \centerline{
      \begin{math}
        f =
        \begin{pmatrix}
          f_0 \\
          f_1 \\
          f_2 \\
          \vdots \\
          f_{x-1} \\
          f_x
        \end{pmatrix}
      \end{math}}
    \begin{gather*}
      % f_i = \frac{1}{\left(\Delta z\right)^2} \cdot \left[K_{i+\frac{1}{2}}^{n+1,m} \cdot \left(H_{i+1}^{n+1,m} - H_i^{n+1,m}\right) - K_{i-\frac{1}{2}}^{n+1,m} \cdot \left(H_{i}^{n+1,m} - H_{i-1}^{n+1,m}\right)\right] \\[+1mm]
      % + \frac{K_{i+\frac{1}{2}}^{n+1,m} - K_{i-\frac{1}{2}}^{n+1,m}}{\Delta z} - C_i^{n+1,m} \cdot \frac{H_i^{n+1,m} - H_i^n}{\Delta t}
      f_i = \frac{1}{\left(\Delta z\right)^2} \cdot \left[K_{i+\frac{1}{2}}^{n+1,m} \cdot \left(H_{i+1}^{n+1,m} - H_i^{n+1,m}\right) - K_{i-\frac{1}{2}}^{n+1,m} \cdot \left(H_{i}^{n+1,m} - H_{i-1}^{n+1,m}\right)\right] \\[+1mm]
      + \frac{K_{i+\frac{1}{2}}^{n+1,m} - K_{i-\frac{1}{2}}^{n+1,m}}{\Delta z} - \frac{\theta_i^{n+1,m}-\theta_i^{n}}{\Delta t}
    \end{gather*}}
\end{frame}

\subsection{Code-Beispiel}
\begin{frame}
  \only<1>{
    \centerline{
      \begin{minipage}{0.75\textwidth}
        \includegraphics[width=1\textwidth]{Grafiken/octave01_beschnitten} \\
        \mycaption{Abbildung: Beispiel für die Lösung eines Gleichungssystems mithilfe von GNU Octave (links: Quelltext, rechts: Octave-Konsole).}
      \end{minipage}}}
  \only<\thefirstElement>{
    \centerline{
      \begin{minipage}{0.75\textwidth}
        \includegraphics[width=1\textwidth]{Grafiken/octave02_beschnitten} \\
        \mycaption{Abbildung: Beispiel für die Lösung eines Gleichungssystems mithilfe von GNU Octave (links: Quelltext, rechts: Octave-Konsole).}
      \end{minipage}}}
  \only<\thesecondElement>{
    \centerline{
      \begin{minipage}{0.75\textwidth}
        \includegraphics[width=1\textwidth]{Grafiken/octave03_beschnitten} \\
        \mycaption{Abbildung: Beispiel für die Lösung eines Gleichungssystems mithilfe von GNU Octave (links: Quelltext, rechts: Octave-Konsole).}
      \end{minipage}}}
  \only<\thethirdElement>{
    \centerline{
      \begin{minipage}{0.75\textwidth}
        \includegraphics[width=1\textwidth]{Grafiken/octave04_beschnitten} \\
        \mycaption{Abbildung: Beispiel für die Lösung eines Gleichungssystems mithilfe von GNU Octave (links: Quelltext, rechts: Octave-Konsole).}
      \end{minipage}}}
\end{frame}

\section{Zusammenfassung}
\begin{frame}[t]
  \centerline{
    \begin{minipage}{0.7\textwidth}
      \visible<1->{
        \includegraphics[width=1.0\textwidth]{Grafiken/IMG_1181_beschnitten} \\
        \mycaption{Abbildung: Beispiel für das Landnutzungsschema der Untersuchungsflächen (hier: KUP und Grünland)
          (Quelle: Falk Richter, verändert).}}
    \end{minipage}}
  \begin{itemize}
  \item<\thefirstElement-> Ziel: \\
    Modellierung des Bodenwasserhaushalts von Ackerkulturen in Nachbarschaft zu KUPs.
  \item<\thesecondElement-> Lösungsansatz:
    \begin{itemize}
    \item<\thesecondElement-> räumlich kleinskalige Modellierung \\
      $\boldsymbol\Rightarrow$ Erfassung von Randeffekten
    \item<\thesecondElement-> implizite numerische Lösung (finite Differenzen + Picard-Iteration) \\
      $\boldsymbol\Rightarrow$ komplexere Implementierung aber stabilere Ergebnisse als explizite Lösung
    \end{itemize}
  \end{itemize}
\end{frame}

\addtocounter{framenumber}{-1} %Reduktion der Folienzahl (-1 für Danksagung und Quellen)

% Festlegen des Absatzeinzugs für die folgenden Seiten (Quellen):
\setlength\parindent{0.5em}
\setlength\baselineskip{0pt}

\section*{}

\begin{frame}[plain]
  \begin{center}
    \colorbox{boxes}{\parbox[c][2cm]{0.75\textwidth}
      {\begin{center}
          \textcolor{letters}{\textbf{\huge Vielen Dank \\ {\small für die Aufmerksamkeit.}}}
        \end{center}
      }
    }
  \end{center} 
  \vspace{1.0cm}
  \begin{block}{Quellen}
    \begin{tiny}
      [1] \textit{Celia Michael A., Bouloutas Efthimios T. \& Zarba Rebecca L.} (1990): A General Mass-Conservative Numerical Solution for the Unsaturated Flow Equation. \textit{Water Resources Research}, 26: 1483-1496.

      [2] \textit{de Jong van Lier Quirijn, Metselaar Klaas \& van Dam Jos C.} (2006): Root Water Extraction and Limiting Soil Hydraulic Conditions Estimated by Numerical Solution. \textit{Vadose Zone Journal}, 5: 1264-1277.
    \end{tiny}
  \end{block}
\end{frame}

\end{document}

%%% Local Variables:
%%% mode: latex
%%% TeX-master: t
%%% End:
