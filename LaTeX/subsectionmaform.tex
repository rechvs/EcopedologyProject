At each time level, equation \eqref{eq:celia_17} has to be solved iteratively for each spatial level.  Additionally, 2 separate scenarios will be simulated, called ``constant pressure'', or cp,  and ``no flux'', or nf, respectively.  The scenarios differ in the treatment of the bottom node.  The cp scenario assumes a constant pressure head at the bottom node, whereas the nf scenario assumes no flux at the bottom node.  All other nodes are treated the same way in both scenarios.

To make full use of the computational capabilities of GNU Octave, equation \eqref{eq:celia_17} needs to be reformulated in matrix form as \parencite{lier_root_2006}
\begin{equation*}
  \begin{bmatrix}
    \beta_1 & \gamma_1 & & & & & \\
    \alpha_2 & \beta_2 & \gamma_2 & & & & \\
    & \alpha_3 & \beta_3 & \gamma_3 & & & \\
    & & & & \alpha_{p-1} & \beta_{p-1} & \gamma_{p-1} \\
    & & & & & \alpha_p & \beta_p
  \end{bmatrix}
  \begin{bmatrix}
    &\delta_1^m \\
    &\delta_2^m \\
    &\delta_3^m \\
    &\delta_{p-1}^m \\
    &\delta_p^m
  \end{bmatrix} =
  \begin{bmatrix}
    &f_1 \\
    &f_2 \\
    &f_3 \\
    &f_{p-1} \\
    &f_p
  \end{bmatrix} \, .
\end{equation*}
The individual elements of the afforementioned matrix and vectors are defined as follows:
\begin{enumerate}
\item for $i = 1$ (bottom spatial level):
  \begin{enumerate}
  \item cp scenario:
    \begin{equation}
    \label{eq:beta_bottom}
    \beta_i = 1
  \end{equation}
  \begin{equation}
    \label{eq:gamma_bottom}
    \gamma_i = 0
  \end{equation}
  \begin{equation}
    \label{eq:f_bottom}
    f_i = 0
  \end{equation}
\end{enumerate}
  \begin{enumerate}
  \item nf scenario:
    \begin{equation}
      \label{eq:beta_bottom}
      \beta_i = \frac{C_i^{n+1,m}}{\Delta t} + \frac{K_{i+\frac{1}{2}}^{n+1,m}}{\del{ \Delta z }^2}
      % \beta_i = 1
  \end{equation}
  \begin{equation}
    \label{eq:gamma_bottom}
    \gamma_i = - \frac{K_{i+\frac{1}{2}}^{n+1,m}}{\del{ \Delta z }^2}
  \end{equation}
  \begin{equation}
    \label{eq:f_bottom}
    f_i = \frac{1}{\del{ \Delta z }^2} K_{i+\frac{1}{2}}^{n+1,m} \del{ H_{i+1}^{n+1,m} - H_{i}^{n+1,m}} + \frac{K_{i+\frac{1}{2}}^{n+1,m}}{\Delta z} - \frac{\theta_i^{n+1,m} - \theta_i^{n}}{\Delta t}
  \end{equation}
\end{enumerate}
\item for $1 < i < p$ (intermediate spatial levels):
  \begin{equation}
    \label{eq:alpha_intermediate}
    \alpha_i = - \frac{K_{i-\frac{1}{2}}^{n+1,m}}{\del{ \Delta z }^2}
  \end{equation}
  \begin{equation}
    \label{eq:beta_intermediate}
    \beta_i = \frac{C_i^{n+1,m}}{\Delta t} + \frac{K_{i+\frac{1}{2}}^{n+1,m}}{\del{ \Delta z }^2} + \frac{K_{i-\frac{1}{2}}^{n+1,m}}{\del{ \Delta z }^2}
  \end{equation}
  \begin{equation}
    \label{eq:gamma_intermediate}
    \gamma_i = - \frac{K_{i+\frac{1}{2}}^{n+1,m}}{\del{ \Delta z }^2}
  \end{equation}
  \begin{equation}
    \label{eq:f_intermediate}
    \begin{split}
      f_i = &\frac{1}{\del{ \Delta z }^2} \del{ K_{i+\frac{1}{2}}^{n+1,m} \del{ H_{i+1}^{n+1,m} - H_{i}^{n+1,m} } - K_{i-\frac{1}{2}}^{n+1,m} \del{ H_{i}^{n+1,m} - H_{i-1}^{n+1,m} } } \\
      &+ \frac{K_{i+\frac{1}{2}}^{n+1,m} - K_{i-\frac{1}{2}}^{n+1,m}}{\Delta z} - \frac{\theta_i^{n+1,m} - \theta_i^{n}}{\Delta t}
    \end{split}
  \end{equation}
\item for $i = p$ (top spatial level):
  \begin{equation}
    \label{eq:alpha_top}
    \alpha_i = 0
  \end{equation}
  \begin{equation}
    \label{eq:beta_top}
    \beta_i = 1
  \end{equation}
  \begin{equation}
    \label{eq:f_top}
    f_i = 0
  \end{equation}
\end{enumerate}



%%% Local Variables:
%%% mode: latex
%%% TeX-master: "basefile/Projekt_basefile"
%%% End:
