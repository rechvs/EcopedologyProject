\subsection{Hydraulic redistribution and soil hydrology modelling}

The effect of hydraulic redistribution executed by mature \emph{Acer saccharum} trees extended less than \SI{5}{\metre} from the tree base \parencite{dawson_hydraulic_1993}.  In the case of \emph{Panicum maximum}, the effect extended no more than \SI{0.3}{\metre} from the plant \parencite{sekiya_applying_2011}.  Therefore, modelling the effect of hydraulic redistribution on soil moisture dynamics requires a spatial resolution of a few \si{\metre} to only a few \si{\centi\metre}, depending on donor plant species.  Hydrological models like WaSim \parencite{schulla_hydrologische_1997} or SWAT \parencite{arnold_large_1998} use grid models with cell edge lengths of several \si{\metre} to a few hundred \si{\metre}.  Additionally, their standard versions usually do not include hydraulic redistribution.  Therefore, such models are not a viable option when attempting to model soil hydrology in the presence of hydraulic redistribution.

Several soil hydrology models incorporating hydraulic redistribution have been proposed \parencite[e.g.,][]{gou_groundwater-soil-plant-atmosphere_2014,ryel_hydraulic_2002}.
\textcite{bayala_hydraulic_2008} extended the WaNuLCAS model \parencite{noordwijk_wanulcas_2011} to determine the magnitude of hydraulic lift in \emph{Parkia biglobosa} and \emph{Vitellaria paradoxa}.  According to the authors, however, the model did overestimate the total amount of hydraulically lifted water.  This stresses the fact that soil hydrology is a matter complex enough to justify being modelled on its own, disregarding influences like light and nutrient availability in favor of physically accurate results.  Additionally, none of the models mentioned above take into account the effect of hydraulic lift on water relations of neighboring plants.  To my knowledge, the only attempt at modelling the effect of hydraulic redistribution in an agroforestry system was made by \textcite{coulibaly_crop_2014}, who also extended the WaNuLCAS model.  In this study, the effect of hydraulic redistribution executed by 3 different tree species (\emph{Parkia biglobosa}, \emph{Vitellaria paradoxa}, and \emph{Adansonia digitata}) on biomass and grain yield of sorghum (\emph{Sorghum bicolor}) was examined.  Results suggest that tree root distribution determines whether or not hydraulic redistribution has a net positive or negative effect on sorghum growth.  Adopting this model to our needs, however, will probably require more effort than writing a new standalone model, since we are mainly interested in soily hydrology dynamics, and not above ground biomass development, which is at the heart of the WaNuLCAS model.  The details of such a standalone model will be outlined in the next section.

%%% Local Variables:
%%% mode: latex
%%% TeX-master: "basefile/Projekt_basefile"
%%% End:
