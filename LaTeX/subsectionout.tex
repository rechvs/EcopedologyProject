Implementation of hydraulic lift and root water uptake would require 3 sets of information, namely:
\begin{enumerate}
\item Pressure head gradient between root system and soil.
\item Radial hydraulic conductivity of the root system (including the rhizosphere).
\item Root-soil interface area in a given volume of soil.
\end{enumerate}
The first and second of these factors are, in principle, comparable to the terms $\nabla \del{ h + z }$ and $K$ in equation \eqref{eq:dareq}, respectively. Supplying these 2 sets of information would thus allow us to compute the amount of water transferred between root system and soil per area unit of root-soil interface.  Calculation of the amount of water actuall transferred would then require the third set of information.  This is necessary, because roots, unlike soil, cannot be considered ubiquitous in the context of soil hydrology modelling.  Instead, root distribution generally follows an exponential curve.  This means that the majority of roots, and therefore of root-soil interface area, is located in relatively shallow depths.  The matter is further complicated by the fact that radial hydraulic conductivity of roots is highly variable within a root system and influenced by factors such as root age, condition of the rhizosphere, or the existence of air gaps between roots and soil.

%%% Local Variables:
%%% mode: latex
%%% TeX-master: "basefile/Projekt_basefile"
%%% End:
