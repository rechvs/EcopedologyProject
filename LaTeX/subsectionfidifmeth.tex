The finite difference method employs a Taylor series in order to solve a partial differential equation \parencite{fornberg_finite_2011}.  Truncating the Taylor series given by equation \eqref{eq:tayse} after the first derivative (i.e., $o = 1$) gives
\begin{align*}
  f(x) &= \frac{(x - x_0)^0}{0!}f^{(0)}(x_0) + \frac{(x - x_0)^1}{1!}f^{(1)}(x_0) + R_1 \\
       &= f(x_0) + (x - x_0)f'(x_0) + R_1
\end{align*}
Solving for $f'(x_0)$ gives
\begin{align*}
  f'(x_0) = \frac{f(x)}{x - x_0} - \frac{f(x_0)}{x - x_0} - \frac{R_1}{x - x_0} \, .
\end{align*}
Assuming that $\frac{R_1}{x - x_0}$ is sufficiently small we obtain
\begin{equation}
  \label{eq:fidifmeth}
  f'(x_0) \approx \frac{f(x) - f(x_0)}{x - x_0} \, .
\end{equation}

Equation \eqref{eq:celia_16} contains 3 terms requiring spatial discretization, namely
\begin{equation*}
  \pd{}{z} \del{ K^{n+1,m} \pd{\delta^m}{z} } \, , \pd{}{z} \del{ K^{n+1,m} \pd{H^{n+1,m}}{z} } \, , \, \text{ and } \, \, \pd{K^{n+1,m}}{z} \, .
\end{equation*}
As an example, I will explain spatial discretization of the second term above.  Spatial discretization first requires assignment of a spatial level, denoted here by subscript $i$.  We thus obtain
\begin{equation*}
  \frac{\Delta}{\Delta z} \del{ K^{n+1,m}\frac{\Delta H^{n+1,m}}{z} }_i \, .
\end{equation*}
Expansion of the outer $\frac{\Delta}{\Delta z}$-term yields
\begin{equation*}
  \frac{1}{\Delta z} \del{ K_{i+\frac{1}{2}}^{n+1,m}\frac{\Delta H_{i+\frac{1}{2}}^{n+1,m}}{\Delta z} - K_{i-\frac{1}{2}}^{n+1,m}\frac{\Delta H_{i-\frac{1}{2}}^{n+1,m}}{\Delta z} } \, .
\end{equation*}
Expansion of the inner $\frac{\Delta}{\Delta z}$-terms yields the spatially discretized version
\begin{align*}
  &\frac{1}{\Delta z} \del{ K_{i+\frac{1}{2}}^{n+1,m}\frac{1}{\Delta z} \del{ H_{i+1}^{n+1,m} - H_{i}^{n+1,m} } - K_{i-\frac{1}{2}}^{n+1,m}\frac{1}{\Delta z} \del{ H_{i}^{n+1,m} - H_{i-1}^{n+1,m} } } \\
  = &\frac{1}{\del{ \Delta z }^2} \del{ K_{i+\frac{1}{2}}^{n+1,m} \del{ H_{i+1}^{n+1,m} - H_{i}^{n+1,m} } - K_{i-\frac{1}{2}}^{n+1,m} \del{ H_{i}^{n+1,m} - H_{i-1}^{n+1,m} } } \, .
\end{align*}
Substituting the spatially discretized versions of all 3 terms into equation \eqref{eq:celia_16} yields \parencite{celia_general_1990}
\begin{equation}
  \label{eq:celia_17}
  \begin{split}
    &C_i^{n+1,m} \frac{\delta_i^m}{\Delta t} - \frac{1}{\del{ \Delta z }^2} \del{ K_{i+\frac{1}{2}}^{n+1,m} \del{ \delta_{i+1}^{m} - \delta_{i}^{m} } - K_{i-\frac{1}{2}}^{n+1,m} \del{ \delta_{i}^{m} - \delta_{i-1}^{m} } } \\
    = &\frac{1}{\del{ \Delta z }^2} \del{ K_{i+\frac{1}{2}}^{n+1,m} \del{ H_{i+1}^{n+1,m} - H_{i}^{n+1,m} } - K_{i-\frac{1}{2}}^{n+1,m} \del{ H_{i}^{n+1,m} - H_{i-1}^{n+1,m} } } \\
    &+ \frac{K_{i+\frac{1}{2}}^{n+1,m} - K_{i-\frac{1}{2}}^{n+1,m}}{\Delta z} - \frac{\theta_i^{n+1,m} - \theta_i^{n}}{\Delta t} \, ,
  \end{split}
\end{equation}
where $K_{i\pm\frac{1}{2}}$ is defined as the arithmetic mean between $K_i$ and $K_{i\pm1}$.  The left side of equation \eqref{eq:celia_17} is now a measure of the amount by which the $m$th iterate fails to satisfy the finite difference approximation \parencite{celia_general_1990}.

%%% Local Variables:
%%% mode: latex
%%% TeX-master: "basefile/Projekt_basefile"
%%% End:
