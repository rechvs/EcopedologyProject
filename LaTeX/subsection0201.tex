\subsection{Hydraulic redistribution}

The term ``hydraulic redistribution'' describes the phenomenon of water movement from moist soil via plant roots into drier soil \parencite{burgess_redistribution_1998}.  It has been documented in several species, such as
\emph{Artemisia tridentata} \parencite{richards_hydraulic_1987},
\emph{Vitellaria paradoxa}, and \emph{Parkia biglobosa} \parencite{bayala_hydraulic_2008},
\emph{Solanum lycopersicum} \parencite{bormann_moisture_1957},
\emph{Eucalyptus camaldulensis}, and \emph{Eucalyptus platypus} \parencite{burgess_tree_2001},
\emph{Medicago sativa} \parencite{corak_water_1987},
\emph{Acer saccharum} \parencite{dawson_hydraulic_1993},
\emph{Populus euphratica} \parencite{hao_hydraulic_2010}
\emph{Bromus tectorum} \parencite{leffler_hydraulic_2005},
\emph{Acacia tortilis} \parencite{ludwig_hydraulic_2003},
\emph{Prosopis velutina} \parencite{scott_ecohydrologic_2008},
\emph{Panicum maximum}, and \emph{Festuca arundinaceum} \parencite{sekiya_applying_2011},
\emph{Zea mays} \parencite{wan_hydraulic_2000},
\emph{Pseudotsuga menziesii}, and \emph{Pinus ponderosa} \parencite{warren_hydraulic_2007}, and
\emph{Quercus petraea} \parencite{zapater_evidence_2011}.

It is considered a passive process driven by a water potential gradient between soil regions of differing water content, which are connected via a plant’s root system \parencite{scott_ecohydrologic_2008}.  

% \parencite{caldwell_hydraulic_1989}

%%% Local Variables:
%%% mode: latex
%%% TeX-master: "basefile/Projekt_basefile"
%%% End:
