The model was run for both scenarios and several different boundary conditions settings (bcs).  In total, 5 model runs were successfully conducted: 3 using the cp scenario, 2 using the nf scenario.  The cp scenario assumed a constant pressure head at the bottom node ($H_1$) whereas the nf scenario assumed no flux at the bottom node.  The parameters settings applying to all model runs are reported in table \ref{tab:moset_ge}.  Note that in all model runs a constant pressure head at the top spatial level ($H_p$) was assumed.  The values of $H_1$ and $H_p$ used in the different boundary conditions settings are reported in table \ref{tab:boundconsettings}.  The initial condition in all model runs was a linear decrease in pressure head from the bottom ($i = 1$) to the top ($i = p$) using the respective boundary conditions.  

\begin{table}[h]
  \centering
  \begin{tabu}{l l}
    \\ \toprule
    Parameter & Value \\
    \midrule
    $\omega$ & $\num{0.26437}$ \\
    $\delta_{th}$ & $0.001 \si{\centi\meter}$ \\
    $\Delta t$ & $\SI{10}{\second}$ \\
    $\Delta z$ & $\SI{0.3}{\centi\meter}$ \\
    $\theta_r$ & $\SI{0}{\cubic\centi\meter\per\cubic\centi\meter}$ \\
    $\theta_s$ & $\SI{0.3879}{\cubic\centi\meter\per\cubic\centi\meter}$ \\
    $\lambda$ & $\num{-0.594}$ \\
    $H_p$ & constant \\
    $K_{sat}$ & $\num{5.926976} \cdot 10^{-3} \si{\centi\meter\per\second}$ \\
    $t_{tot} $ & $\SI{43200}{\second} = \SI{12}{\hour}$ \\
    $v$ & $\num{1.35154}$ \\
    $z_{tot}$ & $\SI{30}{\centi\meter}$ \\
    \bottomrule
  \end{tabu}
  \caption{Parameter settings valid for all model runs.  Parameters needed for equations \eqref{eq:vange_K}, \eqref{eq:vange_theta}, and \eqref{eq:vange_C} were taken from \textcite{renger_ergebnisse_2009}.}
  \label{tab:moset_ge}
\end{table}

\begin{table}[h]
  \centering
  \begin{tabu}{l l l l}
    \\ \toprule
    Parameter & \multicolumn{3}{c}{Value [\si{\centi\meter}]} \\
    & bcs 1 & bcs 2 & bcs 3 \\
    \midrule
    $H_1$ & $0$ & $0$ & $0$ \\
    $H_p$ & $-z_{tot}$ & $-10$ & $-300$ \\
    \bottomrule
  \end{tabu}
  \caption{Values of $H_1$ and $H_p$ used in the different boundary conditions settings.  Note that for the nf scenario, $H_1$ is only valid as an initial condition.  \\bcs: boundary condition setting}
  \label{tab:boundconsettings}
\end{table}

\subsection{Boundary conditions setting 1}

Boundary conditions setting 1 was tailored to impose an equilibrium state as the initial condition.  Under this condition, the pressure head $h$ decreased by the same amount by which the gravitational head $z$ (which is equal to the height above reference level), increased.  Thus, the total hydraulic head was \SI{0}{\centi\meter} throughout the domain.  This bcs served to check whether the model produced realistic results, meaning that no changes should occur under it for both scenarios.  The results for the cp and the nf scenario are reported in figures \ref{fig:cp1} and \ref{fig:nf1}, respectively.

The model produced realistic results.  Since outside influences were absent, it did not leave the equilibrium state of the initial condition.  The value of $H$ was stable for all spatial levels and all time levels in both scenarios.

\begin{figure}[H]
  \centering
  \includegraphics[width=1.0\textwidth]{../../Grafiken/H_3D_plots/Ss_cp_1_1_}
  \caption{Approximation of pressure head ($H$) over all time levels and spatial levels for the cp scenario, using bcs 1.  Spatial level 1 is located at the bottom of the domain.}
  \label{fig:cp1}
\end{figure}

\begin{figure}[H]
  \centering
  \includegraphics[width=1.0\textwidth]{../../Grafiken/H_3D_plots/Ss_nf_1_1_}
  \caption{Approximation of pressure head ($H$) over all time levels and spatial levels for the nf scenario, using bcs 1.  Spatial level 1 is located at the bottom of the domain.}
  \label{fig:nf1}
\end{figure}

\subsection{Boundary conditions setting 2}

Boundary conditions setting 2 consisted of free water at the bottom spatial level, while the top spatial level was only moderately dry ($H_p = \SI{-10}{\centi\meter}$) and moister than under equilibrium conditions.  The results for the cp scenario are reported in figure \ref{fig:cp2}.  The pressure head decreased over time at the intermediate spatial levels ($1 < i < p$).  This was expected, since the initial condition was too moist for the pressure head to remain stable.  The excess moisture traveled downwards and eventuall exited the domain via the bottom spatial level.

For the nf scenario, however, the model did not converge (i.e., it required more than \SI{100}{} iterative steps for a given time step).  The most probable cause for this was the combination of the nf scenario and an excessivley moist soil profile as the initial condition.  The latter presumably led to downward movement of the excess moisture.  At the same time the nf scenario prevented drainage of excess moisture from the bottom spatial level, which was already at saturation ($H_1 = \SI{0}{\centi\meter}$) at the beginning of the model run.  Thus, $H_1$ reached a positive value.  Equations \eqref{eq:vange_K}, \eqref{eq:vange_theta}, and \eqref{eq:vange_C}, however, are only valid if $h \leq \SI{0}{\centi\meter}$.  Therefore, the model was not able to simulate the combination of the nf scenario with boundary conditions setting 2.  In general, the model is incapable of simulating an excessively moist soil profile from which excess moisture cannot escape fast enough to prevent positive pressure head values.

\begin{figure}[H]
  \centering
  \includegraphics[width=1.0\textwidth]{../../Grafiken/H_3D_plots/Ss_cp_2_1_}
  \caption{Approximation of pressure head ($H$) over all time levels and spatial levels for the cp scenario, using bcs 2.  Spatial level 1 is located at the bottom of the domain.}
  \label{fig:cp2}
\end{figure}

\subsection{Boundary conditions setting 3}

Boundary conditions setting 3 consisted of free water at the bottom spatial level, while the top spatial level was severely dry.  The results for the cp scenario are reported in figure \ref{fig:cp3}.  The domain (except for the bottom spatial level) at the start of the model run was drier than under equlibrium conditions.  At the same time, the bottom spatial level, with its constant pressure head of $\SI{0}{\centi\meter}$, served as an infinite water source.  Consequently, the pressure head in spatial levels adjacent to the bottom gradually increased as water moved upwards, following the steep pressure head gradient between bottom and top spatial level.  At the end of the model run, the wetting front had progressed up to ca. the 80th spatial level or $\SI{24}{\centi\meter}$.

The results for the nf scenario are reported in figure \ref{fig:nf3}.  The scenario differed from the cp scenario in that there was no moisture available in the domain other than what was contained within it at the beginning of the model run.  As with the cp scenario, a wetting front formed and moved upwards.  This wetting front did increase the pressure head for most of the spatial levels affected by it.  However, the first ca. 5 spatial levels from the bottom were negatively affected by the wetting front, i.e., their pressure head was lower after the wetting front had progressed across them.  Overall, the wetting front’s progression was severely slower than in the cp scenario and the increase in pressure head at the spatial levels positively affected was lower.  This was due to the fact that the only moisture available for upward movement was the ``excess'' moisture located at the relatively moist lower spatial levels.  At the end of the model run, the wetting front had progressed up to ca. the 50th spatial level or $\SI{15}{\centi\meter}$.

\begin{figure}[H]
  \centering
  \includegraphics[width=1.0\textwidth]{../../Grafiken/H_3D_plots/Ss_cp_3_1_}
  \caption{Approximation of pressure head ($H$) over all time levels and spatial levels for the cp scenario, using bcs 3.  Spatial level 1 is located at the bottom of the domain.}
  \label{fig:cp3}
\end{figure}

\begin{figure}[H]
  \centering
  \includegraphics[width=1.0\textwidth]{../../Grafiken/H_3D_plots/Ss_nf_3_1_}
  \caption{Approximation of pressure head ($H$) over all time levels and spatial levels for the nf scenario, using bcs 3.  Spatial level 1 is located at the bottom of the domain.}
  \label{fig:nf3}
\end{figure}

%%% Local Variables:
%%% mode: latex
%%% TeX-master: "basefile/Projekt_basefile"
%%% End:
