% Soil water is an important aspect of any ecosystem and, apart from light and nutrients, a major governing factor of plant growth.
% Understanding the mechanisms behind water movement in the soil can in turn help to better predict plant growth and crop yields under different climate scenarios.
Movement of water in the soil is generally explained with the help of the of water potential concept.
According to this concept, water flow follows a gradient in water potential, i.e., water will flow spontaneously from regions of higher to regions of lower water potential.
Water potential is in turn influenced by different factors.
In temperate ecosystems the most notable soil-borne of these factors are vertical distance between the soil regions in consideration and their respective water content.
Thus, when considering only the soils of temperate ecosystems, the potential concept can be summarized like this:  water will flow spontaneously from vertically higher to lower and from wetter to drier soil regions.
However, this summary only holds in the absence of plants.
In the context of water potential, a plant acts as a canal linking soil and atmospheric water.
As long as stomatas are opened and a continuous water column between leaves and roots exists, the transpirational demand of the atmoshpere is translated by the plant into a water potential in the rhizosphere, i.e., the soil in close proximity of the roots.
Since in temperate regions the atmosphere’s relative humidity during the vegetation period can be significantly lower than the soil water content, water potential in the rhizosphere can also be drastically lower than in the surrounding soil.
Thus, the presence of plants may fundamentally alter a soil’s water regime, possibly reversing a downward into an upward water flow.
While the influence of a single plant on the water regime of its surrounding soil is well understood, relativley little is known about whether and how neighboring plants influence each other’s water relations.
At the same time, agroforestry systems are considered a promising approach to meet resource and nutrition needs of Earth’s growing population and consequently receive increasing attention.
A topic of particular interest in this regard is ``hydraulic redistribution''.
This term refers to the phenomenon of movement of water from wetter to drier soil regions via plant roots during periods of stomatal closure.
Such a mechanism could have a significant impact on the overall productivity of an agroforestry system, e.g., by effectively providing groundwater access to shallow-rooting highly productive plants via deeply rooted plants.
A more thorough understanding of such water related plant-plant interactions may in turn improve understanding and planning of agroforestry systems in the context of climate change.

This term paper provides a short literature review on hydraulic redistribution and its implementation in soil hydrology modelling.
Subsequently, it introduces a simple 1D vertical soil hydrology model.
The model’s physical and mathematical approach are explained and the results of several model runs are presented, with special emphasis given to the model’s current limitations.
A short conclusion will then summarize the keypoints of the paper’s findings.
% As a first step towards a better understanding of soil water relations in agroforestry systems, the paper presents a 1D vertical soil hydrology model.
% As of yet, the model does not contain a sink term, i.e., water extraction through plant roots cannot be simulated. 

%%% Local Variables:
%%% mode: latex
%%% TeX-master: "basefile/Projekt_basefile"
%%% End:
