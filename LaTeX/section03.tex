In the model described here, water movement is explained with the help of the Richards equation \parencite{richards_capillary_1931}.  The Richards equation is the result of combining Darcy’s law with the continuity principle \parencite{hillel_environmental_1998}.  According to Darcy’s law, water flux $q$ through a given (isotropic) volume of soil is proportional to the hydraulic head drop $\nabla \del{ h + z }$ between inflow and outflow boundary, with hydraulic conductivity $K$ acting as the proportionality factor linking both terms:
\begin{equation}
  \label{eq:dareq}
  q = K(h) \nabla \del{ h + z }
\end{equation}
The continuity principle states that the temporal change in water content of a given volume of soil $\frac{\partial \theta}{\partial t}$ must be equal to the spatial change in water flux $\nabla \cdot q$ from/to this volume:
\begin{equation}
  \label{eq:contprinc}
  \pd{\theta}{t}  = \nabla \cdot q
\end{equation}
Replacing $q$ in equation (\ref{eq:contprinc}) with the right hand side of equation (\ref{eq:dareq}) gives the Richard equation:
\begin{equation}
  \label{eq:richeq}
  \pd{\theta}{t} = \nabla \cdot \del{ K(h) \nabla \del{ h + z } }
\end{equation}
Equation (\ref{eq:richeq}) can be further simplified by applying the vector differential operator $\nabla$: % cp. Notes on Celia et al.1990, p. 24 ff.
\begin{align*}
  \pd{\theta}{t} &= \nabla \cdot \del{ K(h) \nabla \del{ h + z } } \\
                                     &= \nabla \cdot\del{ K(h) \del{ \nabla h + \nabla z } } \\
                                     &= \nabla \cdot\del{ K(h) \del{ \nabla h + \del{ \pd{z}{x},\pd{z}{y},\pd{z}{z} } } } \\
                                     &= \nabla \cdot\del{ K(h) \del{ \nabla h + \del{ 0,0,1 } } } \\
                                     &= \nabla \cdot \del{ K(h) \nabla h + K(h) \del{ 0,0,1 } } \\
                                     &= \nabla \cdot \del{ K(h) \nabla h + \del{ 0,0,K(h) } } \\
                                     &= \nabla \cdot \del{ K(h) \nabla h } + \nabla \cdot \del{ 0,0,K(h) } \\
                                     &= \nabla \cdot \del{ K(h) \nabla h } + \del{ \pd{0}{x},\pd{0}{y},\pd{K(h)}{z}} \\
                                     &= \nabla \cdot \del{ K(h) \nabla h } + \pd{K(h)}{z}
\end{align*}
The Richards equation is thus
\begin{equation}
  \label{eq:richeq_simple}
  \pd{\theta}{t} = \nabla \cdot \del{ K(h) \nabla h } + \pd{K(h)}{z} \, ,
\end{equation}
or
\begin{equation}
  \label{eq:richeq_pde}
  \pd{\theta}{t} - \nabla \cdot \del{ K(h) \nabla h } - \pd{K(h)}{z} = 0 \, .
\end{equation}
% Equation (\ref{eq:richeq_pde}) is also called the ``mixed form'' of the Richards equation, since it is a mixture of the ``$h$-based'' and the ``$\theta$-based'' form \parencite{celia_general_1990}.
The relationships between $\theta$ and $h$ on the one hand, and $K$ and $h$ on the other hand are nonlinear \parencite{celia_general_1990}.  Therefore, numerical approximation is a widely used approach for solving equation (\ref{eq:richeq_pde}).  Discretization of the spatial domain can be achieved by implementing the finite difference method, which itself is based on the Taylor series.

\subsubsection{Taylor series}
A Taylor series is a series expansion of a function about a point \parencite{Weisstein2017}.
% A one-dimensional Taylor series expansion of a real function $f\del{ x }$ about a point $x = a$ is given by
% \begin{equation*}
  % f\del{ x }=f\del{ a }+f'\del{ a }\del{ x-a }+\frac{f''\del{ a }}{2!}\del{ x-a }^2+\frac{f'''\del{ a }}{2!}\del{ x-a }^3+\mathellipsis+\frac{f^{\del{ n }}\del{ a }}{n!}\del{ x-a }^n+\mathellipsis
% \end{equation*}
The Taylor series of a function $f\del{ x }$ can be derived by first noting that the integral of the $\del{ n+1 }$st derivative $f^{\del{ n+1 }}$ of $f\del{ x }$ from the point $x_0$ to an arbitrary point $x_0 + \Delta x = x$ is given by
\begin{align*}
  \int_{x_0}^x{f^{\del{ n+1 }}\del{ x }}\dif x &= \sbr{ f^{\del{ n }}\del{ x } }_{x_0}^x \\
                               &= f^{\del{ n }}\del{ x } - f^{\del{ n }}\del{ x_0 } \, .
\end{align*}
A second integration gives
\begin{align*}
  \int_{x_0}^x\int_{x_0}^x{f^{\del{ n+1 }}\del{ x }} \del{ \dif x }^2 &= \int_{x_0}^x{\del{ f^{\del{ n }}\del{ x } - f^{\del{ n }}\del{ x_0 } }}\dif x \\
                                    &= \int_{x_0}^x{f^{\del{ n }}\del{ x }}\dif x - \int_{x_0}^x{f^{\del{ n }}\del{ x_0 }}\dif x \\
                                    &= \sbr{ f^{\del{ n-1 }}\del{ x } }_{x_0}^x - \sbr{ x f^{\del{ n }}\del{ x_0 } }_{x_0}^x \\
                                    &= f^{\del{ n-1 }}\del{ x } - f^{\del{ n-1 }}\del{ x_0 } - \del{ x f^{\del{ n }}\del{ x } - x_0 f^{\del{ n }}\del{ x_0 } } \\
                                    &= f^{\del{ n-1 }}\del{ x } - f^{\del{ n-1 }}\del{ x_0 } - \del{ x - x_0 } f^{\del{ n }}\del{ x_0 } \, .
\end{align*}
A third integration gives
\begin{align*}
  &\int_{x_0}^x\int_{x_0}^x\int_{x_0}^x{f^{\del{ n+1 }}\del{ x }} \del{ \dif x }^3 \\
  = &\int_{x_0}^x{\del{ f^{\del{ n-1 }}\del{ x } - f^{\del{ n-1 }}\del{ x_0 } - \del{ x - x_0 } f^{\del{ n }}\del{ x_0 } }}\dif x \\
  = &\int_{x_0}^x{f^{\del{ n-1 }}\del{ x }}\dif x - \int_{x_0}^x{f^{\del{ n-1 }}\del{ x_0 }}\dif x - \int_{x_0}^x\del{ x - x_0 }{f^{\del{ n }}\del{ x_0 }}\dif x \\
  = &\int_{x_0}^x{f^{\del{ n-1 }}\del{ x }}\dif x - \int_{x_0}^x{f^{\del{ n-1 }}\del{ x_0 }}\dif x - \int_{x_0}^x\del{x f^{\del{ n }}\del{ x_0 } - x_0 f^{\del{ n }}\del{ x_0 }}\dif x \\
  = &\int_{x_0}^x{f^{\del{ n-1 }}\del{ x }}\dif x - \int_{x_0}^x{f^{\del{ n-1 }}\del{ x_0 }}\dif x - \del{ \int_{x_0}^x{x f^{\del{ n }}\del{ x_0 }}\dif x - \int_{x_0}^x{x_0 f^{\del{ n }}\del{ x_0 }}\dif x } \\
  = &\sbr{ f^{\del{ n-2 }}\del{ x } }_{x_0}^x -\sbr{ x f^{\del{ n-1 }}\del{ x_0 } }_{x_0}^x - \del{ \sbr{ \frac{1}{2}x^2f^{\del{ n }}\del{ x_0 } }_{x_0}^x - \sbr{ xx_0f^{\del{ n }}\del{ x_0 } }_{x_0}^x } \\
  = &f^{\del{ n-2 }}\del{ x }
    - f^{\del{ n-2 }}\del{ x_0 }
    - \del{ x f^{\del{ n-1 }}\del{ x_0 }
    - x_0 f^{\del{ n-1 }}\del{ x_0 } } \\
  &- \del{ \del{ \frac{1}{2} x^2 f^{\del{ n }}\del{ x_0 }
    - \frac{1}{2} x_0^2 f^{\del{ n }}\del{ x_0 } }
    - \del{ x x_0 f^{\del{ n }}\del{ x_0 }
    - x_0 x_0 f^{\del{ n }}\del{ x_0 } } } \\
  = &f^{\del{ n-2 }}\del{ x }
    - f^{\del{ n-2 }}\del{ x_0 }
    - \del{ x - x_0 } f^{\del{ n-1 }}\del{ x_0 } - \del{ \frac{1}{2} x^2 - \frac{1}{2} x_0^2 - x x_0 + x_0^2 } f^{\del{ n }}\del{ x_0 } \\
  = &f^{\del{ n-2 }}\del{ x }
    - f^{\del{ n-2 }}\del{ x_0 }
    - \del{ x - x_0 } f^{\del{ n-1 }}\del{ x_0 } - \del{ \frac{1}{2} x^2 - x x_0 + \frac{1}{2} x_0^2 } f^{\del{ n }}\del{ x_0 } \\
  = &f^{\del{ n-2 }}\del{ x }
    - f^{\del{ n-2 }}\del{ x_0 }
    - \del{ x - x_0 } f^{\del{ n-1 }}\del{ x_0 } - \frac{1}{2} \del{ x^2 - 2 x x_0 + x_0^2 } f^{\del{ n }}\del{ x_0 } \\
  = &f^{\del{ n-2 }}\del{ x }
    - f^{\del{ n-2 }}\del{ x_0 }
    - \del{ x - x_0 } f^{\del{ n-1 }}\del{ x_0 } - \frac{1}{2} \del{ x - x_0 }^2 f^{\del{ n }}\del{ x_0 } \\
  = &f^{\del{ n-2 }}\del{ x }
    - f^{\del{ n-2 }}\del{ x_0 }
    - \del{ x - x_0 } f^{\del{ n-1 }}\del{ x_0 } - \frac{\del{ x - x_0 }^2}{2 !}  f^{\del{ n }}\del{ x_0 } \, .
\end{align*}
Continuing up to $n + 1$ integrations gives
\begin{align*}
  &\underbrace{\int \hspace{-4mm} \hdots \hspace{-1mm} \int_{x_0}^x}_{n+1}{f^{\del{ n+1 }}\del{ x }\del{ \dif x }^{\del{ n+1 }}} \\
  = &f\del{ x } - f\del{ x_0 } - \del{ x - x_0 } f'\del{ x_0 } - \frac{\del{ x - x_0 }^2}{2!} f''\del{ x_0 } - \frac{\del{ x - x_0 }^3}{3!} f'''\del{ x_0 } \\
  &- \mathellipsis - \frac{\del{ x - x_0 }^n}{n!} f^{\del{ n }}\del{ x_0 } \, .
\end{align*}
Solving for $f\del{ x }$ gives
\begin{align*}
  f\del{ x } = &f\del{ x_0 }
           + \del{ x - x_0 } f'\del{ x_0 }
           + \frac{\del{ x-x_0 }^2}{2!} f''\del{ x_0 } \\
         &+ \frac{\del{ x - x_0 }^3}{3!} f'''\del{ x_0 }
           + \mathellipsis \frac{\del{ x - x_0 }^n}{n!} f^{\del{ n }}\del{ x_0 }
           + \underbrace{\int \hspace{-4mm} \hdots \hspace{-1mm} \int_{x_0}^x}_{n+1}{f^{\del{ n+1 }}\del{ x }\del{ \dif x }^{\del{ n+1 }}} \\
  = &\sum_{k=0}^n{\frac{\del{ x - x_0 }^k}{k!}}f^{\del{ k }}\del{ x_0 } + \underbrace{\int \hspace{-4mm} \hdots \hspace{-1mm} \int_{x_0}^x}_{n+1}{f^{\del{ n+1 }}\del{ x }\del{ \dif x }^{\del{ n+1 }}} \, .
\end{align*}
Noting that the Lagrange remainder $R_n$ is given by
\begin{align*}
  R_n = &\underbrace{\int \hspace{-4mm} \hdots \hspace{-1mm} \int_{x_0}^x}_{n+1}{f^{\del{ n+1 }}\del{ x }\del{ \dif x }^{\del{ n+1 }}} \, ,
\end{align*}
we obtain
\begin{equation}
  \label{eq:tayse}
  f\del{ x } = \sum_{k=0}^n{\frac{\del{ x - x_0 }^k}{k!}}f^{\del{ k }}\del{ x_0 } + R_n \, .
\end{equation}
The right hand side of equation \eqref{eq:tayse} is the one-dimensional Taylor series of function $f\del{ x }$.  $\theta^{n+1,m+1}$ in equation \eqref{eq:celia_14} can be estimated using equation \eqref{eq:tayse} by setting $n = 1, x = H^{n+1,m+1}, x_0 = H^{n+1,m}, f\del{ x } = \theta^{n+1,m+1}, f\del{ x_0 } = \theta^{n+1,m}, f'\del{ x_0 } = \evalat{\od{\theta}{h}}{n+1,m}$ \parencite{celia_general_1990}:
\begin{align}
  \label{eq:celia_15}
  % \theta^{n+1,m+1} = \sum_{k=0}^1\frac{\del{ H^{n+1,m+1} - H^{n+1,m} }^k}{k!}f^{\del{ k }}\del{ H^{n+1,m} } + R_1
  \theta^{n+1,m+1} = &\sum_{k=0}^1\frac{\del{ H^{n+1,m+1} - H^{n+1,m} }^k}{k!}f^{\del{ k }}\del{ H^{n+1,m} } + R_1 \\
  = &\frac{\del{ H^{n+1,m+1} - H^{n+1,m} }^0}{0!} f^{\del{ 0 }}\del{ H^{n+1,m} } + \frac{\del{ H^{n+1,m+1} - H^{n+1,m} }^1}{1!} f^{\del{ 1 }}\del{ H^{n+1,m} } + R_1 \\
                   = &f\del{ H^{n+1,m} } + \del{ H^{n+1,m+1} - H^{n+1,m} } f'\del{ H^{n+1,m} } + 0 \del{ \delta^2 }
\end{align}

%%% Local Variables:
%%% mode: latex
%%% TeX-master: "basefile/Projekt_basefile"
%%% End:


\subsubsection{Finite difference method}
The finite difference method employs a Taylor series in order to solve a partial differential equation like equation \eqref{eq:richeq_1D} \parencite{fornberg_finite_2011}.  Truncating the Taylor series given by equation \eqref{eq:tayse} after the first derivative (i.e., $o = 1$) gives
\begin{align*}
  f(x) &= \frac{(x - x_0)^0}{0!}f^{(0)}(x_0) + \frac{(x - x_0)^1}{1!}f^{(1)}(x_0) + R_1 \\
       &= f(x_0) + (x - x_0)f'(x_0) + R_1
\end{align*}
Solving for $f'(x_0)$ gives
\begin{align*}
  f'(x_0) = \frac{f(x)}{x - x_0} - \frac{f(x_0)}{x - x_0} - \frac{R_1}{x - x_0} \, .
\end{align*}
Assuming that $\frac{R_1}{x - x_0}$ is sufficiently small we obtain
\begin{equation}
  \label{eq:fidifmeth}
  f'(x_0) \approx \frac{f(x) - f(x_0)}{x - x_0} \, .
\end{equation}

Equation \eqref{eq:celia_16} contains 3 terms requiring spatial discretization, namely
\begin{equation*}
  \pd{}{z} \del{ K^{n+1,m} \pd{H^{n+1,m}}{z} } \, , \, \pd{K^{n+1,m}}{z} \, , \, \text{ and } \, \pd{}{z} \del{ K^{n+1,m} \pd{\delta^m}{z} } \, .
\end{equation*}
As an example, I will explain spatial discretization of the first term above.  Spatial discretization first requires assignment of a spatial level, denoted here by $i$.  We thus obtain
\begin{equation*}
  \frac{\Delta}{\Delta z} \del{ K^{n+1,m}\frac{\Delta H^{n+1,m}}{z} }_i \, .
\end{equation*}
Expansion of the outer $\frac{\Delta}{\Delta z}$-term yields
\begin{equation*}
  \frac{1}{\Delta z} \del{ K_{i+\frac{1}{2}}^{n+1,m}\frac{\Delta H_{i+\frac{1}{2}}^{n+1,m}}{\Delta z} - K_{i-\frac{1}{2}}^{n+1,m}\frac{\Delta H_{i-\frac{1}{2}}^{n+1,m}}{\Delta z} } \, .
\end{equation*}
Expansion of the inner $\frac{\Delta}{\Delta z}$-terms yields the spatially discretized version
\begin{align*}
  &\frac{1}{\Delta z} \del{ K_{i+\frac{1}{2}}^{n+1,m}\frac{1}{\Delta z} \del{ H_{i+1}^{n+1,m} - H_{i}^{n+1,m} } - K_{i-\frac{1}{2}}^{n+1,m}\frac{1}{\Delta z} \del{ H_{i}^{n+1,m} - H_{i-1}^{n+1,m} } } \\
  = &\frac{1}{\del{ \Delta z }^2} \del{ K_{i+\frac{1}{2}}^{n+1,m} \del{ H_{i+1}^{n+1,m} - H_{i}^{n+1,m} } - K_{i-\frac{1}{2}}^{n+1,m} \del{ H_{i}^{n+1,m} - H_{i-1}^{n+1,m} } } \, .
\end{align*}

Substituting the spatially discretized versions of all 3 terms into equation \eqref{eq:celia_16} yields \parencite{celia_general_1990}
\begin{equation}
  \label{eq:celia_17}
  \begin{split}
    % \pd{}{z} \del{ K^{n+1,m} \pd{H^{n+1,m}}{z} } + \pd{K^{n+1,m}}{z} - \frac{\theta^{n+1,m} - \theta^n}{\Delta t} = \del{ \frac{1}{\Delta t} C^{n+1,m} } \delta^m - \pd{}{z} \del{ K^{n+1,m} \pd{\delta^m}{z} } \, .
    &\frac{1}{\del{ \Delta z }^2} \del{ K_{i+\frac{1}{2}}^{n+1,m} \del{ H_{i+1}^{n+1,m} - H_{i}^{n+1,m} } - K_{i-\frac{1}{2}}^{n+1,m} \del{ H_{i}^{n+1,m} - H_{i-1}^{n+1,m} } } \\
    &+ \frac{K_{i+\frac{1}{2}}^{n+1,m} - K_{i-\frac{1}{2}}^{n+1,m}}{\Delta z} - \frac{\theta_i^{n+1,m} - \theta_i^{n}}{\Delta t} \\
    = &\del{ \frac{1}{\Delta t} C^{n+1,m} } \delta^m - \frac{1}{\del{ \Delta z }^2} \del{ K_{i+\frac{1}{2}}^{n+1,m} \del{ \delta_{i+1}^{m} - \delta_{i}^{m} } - K_{i-\frac{1}{2}}^{n+1,m} \del{ \delta_{i}^{m} - \delta_{i-1}^{m} } }  \, ,
  \end{split}
\end{equation}
where $K_{i\pm\frac{1}{2}}$ is defined as the arithmetic mean between $K_i$ and $K_{i\pm1}$.  The left side of equation \eqref{eq:celia_17} is now a measure of the amount by which the $m$th iterate fails to satisfy the finite difference approximation \parencite{celia_general_1990}.

%%% Local Variables:
%%% mode: latex
%%% TeX-master: "basefile/Projekt_basefile"
%%% End:



% There exist 3 standard forms of the Richards equation \parencite{celia_general_1990}:  \\
% $h$ based form
% \begin{equation}
  % \label{eq:h_based_richeq}
  % C(h) \frac{\partial h}{\partial t} - \nabla \cdot K(h)\nabla h - \frac{\partial K(h)}{\partial z} = 0
% \end{equation}
% $\theta$ based form
% \begin{equation}
  % \label{eq:theta_base_richeq}
  % \frac{\partial \theta}{\partial t} - \nabla \cdot D(\theta)\nabla \theta - \frac{\partial K(h)}{\partial z} = 0
% \end{equation}
% mixed form
% \begin{equation}
  % \label{eq:mixed_richeq}
  % \frac{\partial \theta}{\partial t} - \nabla \cdot K(h)\nabla h - \frac{\partial K(h)}{\partial z} = 0
% \end{equation}
% The model employs the mixed

% The hydraulic conductivity for a given pressure head is estimated using the equation given by \textcite{genuchten_closed-form_1980}:
% \begin{equation}
  % \label{eq:vange_hycon}
  % K(h) = K_s 
% \end{equation}

%%% Local Variables:
%%% mode: latex
%%% TeX-master: "basefile/Projekt_basefile"
%%% End:
