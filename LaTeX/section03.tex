In the model described here, water movement is explained with the help of the Richards equation \parencite{richards_capillary_1931}.  The Richards equation is the result of combining Darcy’s law with the continuity principle \parencite{hillel_environmental_1998}.  According to Darcy’s law, water flux $q$ through a given (isotropic) volume of soil is proportional to the hydraulic head drop $\nabla \del{ h + z }$ between inflow and outflow boundary, with hydraulic conductivity $K$ acting as the proportionality factor linking both terms:
\begin{equation}
  \label{eq:dareq}
  q = K(h) \nabla \del{ h + z }
\end{equation}
The continuity principle states that the temporal change in water content of a given volume of soil $\frac{\partial \theta}{\partial t}$ must be equal to the spatial change in water flux $\nabla \cdot q$ from/to this volume:
\begin{equation}
  \label{eq:contprinc}
  \pd{\theta}{t}  = \nabla \cdot q
\end{equation}
Replacing $q$ in equation (\ref{eq:contprinc}) with the right hand side of equation (\ref{eq:dareq}) gives the Richard equation:
\begin{equation}
  \label{eq:richeq}
  \pd{\theta}{t} = \nabla \cdot \del{ K(h) \nabla \del{ h + z } }
\end{equation}
Equation (\ref{eq:richeq}) can be further simplified by applying the vector differential operator $\nabla$: % cp. Notes on Celia et al.1990, p. 24 ff.
\begin{align*}
  \pd{\theta}{t} &= \nabla \cdot \del{ K(h) \nabla \del{ h + z } } \\
                                     &= \nabla \cdot\del{ K(h) \del{ \nabla h + \nabla z } } \\
                                     &= \nabla \cdot\del{ K(h) \del{ \nabla h + \del{ \pd{z}{x},\pd{z}{y},\pd{z}{z} } } } \\
                                     &= \nabla \cdot\del{ K(h) \del{ \nabla h + \del{ 0,0,1 } } } \\
                                     &= \nabla \cdot \del{ K(h) \nabla h + K(h) \del{ 0,0,1 } } \\
                                     &= \nabla \cdot \del{ K(h) \nabla h + \del{ 0,0,K(h) } } \\
                                     &= \nabla \cdot \del{ K(h) \nabla h } + \nabla \cdot \del{ 0,0,K(h) } \\
                                     &= \nabla \cdot \del{ K(h) \nabla h } + \del{ \pd{0}{x},\pd{0}{y},\pd{K(h)}{z}} \\
                                     &= \nabla \cdot \del{ K(h) \nabla h } + \pd{K(h)}{z}
\end{align*}
The Richards equation is thus
\begin{equation}
  \label{eq:richeq_3D}
  \pd{\theta}{t} = \nabla \cdot \del{ K(h) \nabla h } + \pd{K(h)}{z} \, .
\end{equation}
The model presented here is a one-dimensional model, considering only the vertical spatial dimension and disregarding $x$ and $y$ directions.  Therefore, equation \eqref{eq:richeq_3D} needs to be adjusted to the one-dimensional case by setting $\pd{f(x,y,z)}{x} = 0$ and $\pd{f(x,y,z)}{y} = 0$:
\begin{align*}
  \pd{\theta}{t} &= \nabla \cdot \del{ K(h) \nabla h } + \pd{K(h)}{z} \\
                 &= \nabla \cdot \del{ K(h) \del{ \pd{h}{x},\pd{h}{y},\pd{h}{z} } } + \pd{K(h)}{z} \\
                 % &= \nabla \cdot \del{ K(h) \del{ 0, 0, \pd{h}{z} } } + \pd{K(h)}{z} \\
                 % &= \nabla \cdot  \del{ 0, 0, K(h) \pd{h}{z} } + \pd{K(h)}{z} \\
                 &= \nabla \cdot \del{ K(h) \pd{h}{x},K(h) \pd{h}{y},K(h) \pd{h}{z} } + \pd{K(h)}{z} \\
                 &= \pd{}{x} \del{ K(h) \pd{h}{x} } + \pd{}{y} \del{ K(h) \pd{h}{y} } + \pd{}{z} \del{ K(h) \pd{h}{z} } + \pd{K(h)}{z} \\
                 &= 0 + 0 + \pd{}{z} \del{ K(h) \pd{h}{z} } + \pd{K(h)}{z} \, .
\end{align*}
The Richards equation for the one-dimensional case considering only the vertical spatial dimension is thus
\begin{equation}
  \label{eq:richeq_1D}
  % \pd{\theta}{t} - \nabla \cdot \del{ K(h) \nabla h } - \pd{K(h)}{z} = 0 \, .
  \pd{\theta}{t} = \frac{\partial}{\partial z} \del{ K(h) \pd{h}{z} } + \pd{K(h)}{z} \, .
\end{equation}
% Equation (\ref{eq:richeq_pde}) is also called the ``mixed form'' of the Richards equation, since it is a mixture of the ``$h$-based'' and the ``$\theta$-based'' form \parencite{celia_general_1990}.
The relationships between $\theta$ and $h$ on the one hand, and $K$ and $h$ on the other hand are nonlinear \parencite{celia_general_1990}.  Therefore, numerical approximation is a widely used approach for solving equation \eqref{eq:richeq_1D}, requiring both temporal discretization (i.e., discretization with respect to $t$) and spatial discretization (i.e., discretization with respect to $z$) of the equation.

In general, 2 approaches for temporally discretizing differential equations exist:  explicit methods and implicit methods.  Explicit methods estimate the value of a function at time $t_{n+1} = t_n + \Delta t$ based on the value at time $t_n$. For example, the Euler forward method uses the equation
\begin{equation}
  \label{eq:euformeth}
  s_{n+1} = s_n + (t_{n+1} - t_n) f(t_n,s_n)
\end{equation}
to obtain the value $s_{n+1}$.  Since the method attempts to calculate the value at $t_{n+1}$ based on information valid at $t_n$ it is prone to error and instable results \parencite{Weisstein2017a}.  Implicit methods do not suffer from these shortcomings.  The model presented here employs the Euler backward method, which is an example of an implicit method \parencite{Weisstein2017b}.

\subsection{Euler backward method}
The Euler backward method is based on the mean value theorem.  For the mean value theorem to be applicable, a function $f(x)$ must be differentiable in the open interval \intoo{t_0,t_1} and continuous in the closed interval \intcc{t_0,t_1} \parencite{Weisstein2017c}.  If these conditions are met, then there is at least one point $t_2$ in \intoo{t_0,t_1} such that
\begin{equation*}
  % f'(t_2) = \frac{f(t_1) - f(t_0)}{t_1 - t_0} \, .
% \end{equation*}
% Rearranging gives
% \begin{equation*}
  f(t_1) = f(t_0) + (t_1 - t_0) f'(t_2) \, .
\end{equation*}

% Let us assume we are given the ordinary differential equation
% \begin{equation*}
  % y'(t) = f(t, y(t)) \, .
% \end{equation*}
% Applying the mean value theorem we obtain
% \begin{equation*}
  % y(t_1) = y(t_0) + (t_1 - t_0) y'(t_2) \, .
% \end{equation*}
The Euler backward method works by assuming that $t_2 = t_1$ \parencite{hairer_solving_2009}, so that
\begin{equation*}
  % y(t_1) = y(t_0) + (t_1 - t_0) y'(t_1) \, .
  f(t_1) = f(t_0) + (t_1 - t_0) f'(t_1) \, .
\end{equation*}
Applying the Euler backward method coupled with the Picard iteration to equation \eqref{eq:richeq_1D}, we obtain \parencite{celia_general_1990}
\begin{equation}
  \label{eq:celia_14}
  \frac{\theta^{n+1,m+1} - \theta^n}{\Delta t} - \frac{\partial}{\partial z} \del{ K^{n+1,m}\pd{H^{n+1,m+1}}{z} } - \pd{K^{n+1,m}}{z} = 0 \, ,
\end{equation}
where $n$ and $m$ denote time level and iteration level, respectively.  The value $\theta^{n+1,m+1}$ in equation \eqref{eq:celia_14} can be approximated using a truncated Taylor series, which will be explained in the following section.


%%% Local Variables:
%%% mode: latex
%%% TeX-master: "basefile/Projekt_basefile"
%%% End:


\subsection{Taylor series}
A Taylor series is a series expansion of a function about a point \parencite{Weisstein2017}.
% A one-dimensional Taylor series expansion of a real function $f\del{ x }$ about a point $x = a$ is given by
% \begin{equation*}
  % f\del{ x }=f\del{ a }+f'\del{ a }\del{ x-a }+\frac{f''\del{ a }}{2!}\del{ x-a }^2+\frac{f'''\del{ a }}{2!}\del{ x-a }^3+\mathellipsis+\frac{f^{\del{ n }}\del{ a }}{n!}\del{ x-a }^n+\mathellipsis
% \end{equation*}
The Taylor series of a function $f\del{ x }$ can be derived by first noting that the integral of the $\del{ n+1 }$st derivative $f^{\del{ n+1 }}$ of $f\del{ x }$ from the point $x_0$ to an arbitrary point $x_0 + \Delta x = x$ is given by
\begin{align*}
  \int_{x_0}^x{f^{\del{ n+1 }}\del{ x }}\dif x &= \sbr{ f^{\del{ n }}\del{ x } }_{x_0}^x \\
                               &= f^{\del{ n }}\del{ x } - f^{\del{ n }}\del{ x_0 } \, .
\end{align*}
A second integration gives
\begin{align*}
  \int_{x_0}^x\int_{x_0}^x{f^{\del{ n+1 }}\del{ x }} \del{ \dif x }^2 &= \int_{x_0}^x{\del{ f^{\del{ n }}\del{ x } - f^{\del{ n }}\del{ x_0 } }}\dif x \\
                                    &= \int_{x_0}^x{f^{\del{ n }}\del{ x }}\dif x - \int_{x_0}^x{f^{\del{ n }}\del{ x_0 }}\dif x \\
                                    &= \sbr{ f^{\del{ n-1 }}\del{ x } }_{x_0}^x - \sbr{ x f^{\del{ n }}\del{ x_0 } }_{x_0}^x \\
                                    &= f^{\del{ n-1 }}\del{ x } - f^{\del{ n-1 }}\del{ x_0 } - \del{ x f^{\del{ n }}\del{ x } - x_0 f^{\del{ n }}\del{ x_0 } } \\
                                    &= f^{\del{ n-1 }}\del{ x } - f^{\del{ n-1 }}\del{ x_0 } - \del{ x - x_0 } f^{\del{ n }}\del{ x_0 } \, .
\end{align*}
A third integration gives
\begin{align*}
  &\int_{x_0}^x\int_{x_0}^x\int_{x_0}^x{f^{\del{ n+1 }}\del{ x }} \del{ \dif x }^3 \\
  = &\int_{x_0}^x{\del{ f^{\del{ n-1 }}\del{ x } - f^{\del{ n-1 }}\del{ x_0 } - \del{ x - x_0 } f^{\del{ n }}\del{ x_0 } }}\dif x \\
  = &\int_{x_0}^x{f^{\del{ n-1 }}\del{ x }}\dif x - \int_{x_0}^x{f^{\del{ n-1 }}\del{ x_0 }}\dif x - \int_{x_0}^x\del{ x - x_0 }{f^{\del{ n }}\del{ x_0 }}\dif x \\
  = &\int_{x_0}^x{f^{\del{ n-1 }}\del{ x }}\dif x - \int_{x_0}^x{f^{\del{ n-1 }}\del{ x_0 }}\dif x - \int_{x_0}^x\del{x f^{\del{ n }}\del{ x_0 } - x_0 f^{\del{ n }}\del{ x_0 }}\dif x \\
  = &\int_{x_0}^x{f^{\del{ n-1 }}\del{ x }}\dif x - \int_{x_0}^x{f^{\del{ n-1 }}\del{ x_0 }}\dif x - \del{ \int_{x_0}^x{x f^{\del{ n }}\del{ x_0 }}\dif x - \int_{x_0}^x{x_0 f^{\del{ n }}\del{ x_0 }}\dif x } \\
  = &\sbr{ f^{\del{ n-2 }}\del{ x } }_{x_0}^x -\sbr{ x f^{\del{ n-1 }}\del{ x_0 } }_{x_0}^x - \del{ \sbr{ \frac{1}{2}x^2f^{\del{ n }}\del{ x_0 } }_{x_0}^x - \sbr{ xx_0f^{\del{ n }}\del{ x_0 } }_{x_0}^x } \\
  = &f^{\del{ n-2 }}\del{ x }
    - f^{\del{ n-2 }}\del{ x_0 }
    - \del{ x f^{\del{ n-1 }}\del{ x_0 }
    - x_0 f^{\del{ n-1 }}\del{ x_0 } } \\
  &- \del{ \del{ \frac{1}{2} x^2 f^{\del{ n }}\del{ x_0 }
    - \frac{1}{2} x_0^2 f^{\del{ n }}\del{ x_0 } }
    - \del{ x x_0 f^{\del{ n }}\del{ x_0 }
    - x_0 x_0 f^{\del{ n }}\del{ x_0 } } } \\
  = &f^{\del{ n-2 }}\del{ x }
    - f^{\del{ n-2 }}\del{ x_0 }
    - \del{ x - x_0 } f^{\del{ n-1 }}\del{ x_0 } - \del{ \frac{1}{2} x^2 - \frac{1}{2} x_0^2 - x x_0 + x_0^2 } f^{\del{ n }}\del{ x_0 } \\
  = &f^{\del{ n-2 }}\del{ x }
    - f^{\del{ n-2 }}\del{ x_0 }
    - \del{ x - x_0 } f^{\del{ n-1 }}\del{ x_0 } - \del{ \frac{1}{2} x^2 - x x_0 + \frac{1}{2} x_0^2 } f^{\del{ n }}\del{ x_0 } \\
  = &f^{\del{ n-2 }}\del{ x }
    - f^{\del{ n-2 }}\del{ x_0 }
    - \del{ x - x_0 } f^{\del{ n-1 }}\del{ x_0 } - \frac{1}{2} \del{ x^2 - 2 x x_0 + x_0^2 } f^{\del{ n }}\del{ x_0 } \\
  = &f^{\del{ n-2 }}\del{ x }
    - f^{\del{ n-2 }}\del{ x_0 }
    - \del{ x - x_0 } f^{\del{ n-1 }}\del{ x_0 } - \frac{1}{2} \del{ x - x_0 }^2 f^{\del{ n }}\del{ x_0 } \\
  = &f^{\del{ n-2 }}\del{ x }
    - f^{\del{ n-2 }}\del{ x_0 }
    - \del{ x - x_0 } f^{\del{ n-1 }}\del{ x_0 } - \frac{\del{ x - x_0 }^2}{2 !}  f^{\del{ n }}\del{ x_0 } \, .
\end{align*}
Continuing up to $n + 1$ integrations gives
\begin{align*}
  &\underbrace{\int \hspace{-4mm} \hdots \hspace{-1mm} \int_{x_0}^x}_{n+1}{f^{\del{ n+1 }}\del{ x }\del{ \dif x }^{\del{ n+1 }}} \\
  = &f\del{ x } - f\del{ x_0 } - \del{ x - x_0 } f'\del{ x_0 } - \frac{\del{ x - x_0 }^2}{2!} f''\del{ x_0 } - \frac{\del{ x - x_0 }^3}{3!} f'''\del{ x_0 } \\
  &- \mathellipsis - \frac{\del{ x - x_0 }^n}{n!} f^{\del{ n }}\del{ x_0 } \, .
\end{align*}
Solving for $f\del{ x }$ gives
\begin{align*}
  f\del{ x } = &f\del{ x_0 }
           + \del{ x - x_0 } f'\del{ x_0 }
           + \frac{\del{ x-x_0 }^2}{2!} f''\del{ x_0 } \\
         &+ \frac{\del{ x - x_0 }^3}{3!} f'''\del{ x_0 }
           + \mathellipsis \frac{\del{ x - x_0 }^n}{n!} f^{\del{ n }}\del{ x_0 }
           + \underbrace{\int \hspace{-4mm} \hdots \hspace{-1mm} \int_{x_0}^x}_{n+1}{f^{\del{ n+1 }}\del{ x }\del{ \dif x }^{\del{ n+1 }}} \\
  = &\sum_{k=0}^n{\frac{\del{ x - x_0 }^k}{k!}}f^{\del{ k }}\del{ x_0 } + \underbrace{\int \hspace{-4mm} \hdots \hspace{-1mm} \int_{x_0}^x}_{n+1}{f^{\del{ n+1 }}\del{ x }\del{ \dif x }^{\del{ n+1 }}} \, .
\end{align*}
Noting that the Lagrange remainder $R_n$ is given by
\begin{align*}
  R_n = &\underbrace{\int \hspace{-4mm} \hdots \hspace{-1mm} \int_{x_0}^x}_{n+1}{f^{\del{ n+1 }}\del{ x }\del{ \dif x }^{\del{ n+1 }}} \, ,
\end{align*}
we obtain
\begin{equation}
  \label{eq:tayse}
  f\del{ x } = \sum_{k=0}^n{\frac{\del{ x - x_0 }^k}{k!}}f^{\del{ k }}\del{ x_0 } + R_n \, .
\end{equation}
The right hand side of equation \eqref{eq:tayse} is the one-dimensional Taylor series of function $f\del{ x }$.  $\theta^{n+1,m+1}$ in equation \eqref{eq:celia_14} can be estimated using equation \eqref{eq:tayse} by setting $n = 1, x = H^{n+1,m+1}, x_0 = H^{n+1,m}, f\del{ x } = \theta^{n+1,m+1}, f\del{ x_0 } = \theta^{n+1,m}, f'\del{ x_0 } = \evalat{\od{\theta}{h}}{n+1,m}$ \parencite{celia_general_1990}:
\begin{align}
  \label{eq:celia_15}
  % \theta^{n+1,m+1} = \sum_{k=0}^1\frac{\del{ H^{n+1,m+1} - H^{n+1,m} }^k}{k!}f^{\del{ k }}\del{ H^{n+1,m} } + R_1
  \theta^{n+1,m+1} = &\sum_{k=0}^1\frac{\del{ H^{n+1,m+1} - H^{n+1,m} }^k}{k!}f^{\del{ k }}\del{ H^{n+1,m} } + R_1 \\
  = &\frac{\del{ H^{n+1,m+1} - H^{n+1,m} }^0}{0!} f^{\del{ 0 }}\del{ H^{n+1,m} } + \frac{\del{ H^{n+1,m+1} - H^{n+1,m} }^1}{1!} f^{\del{ 1 }}\del{ H^{n+1,m} } + R_1 \\
                   = &f\del{ H^{n+1,m} } + \del{ H^{n+1,m+1} - H^{n+1,m} } f'\del{ H^{n+1,m} } + 0 \del{ \delta^2 }
\end{align}

%%% Local Variables:
%%% mode: latex
%%% TeX-master: "basefile/Projekt_basefile"
%%% End:


Discretization of the spatial domain can be achieved by implementing the finite difference method, which will be explained in the following section.

\subsection{Finite difference method}
The finite difference method employs a Taylor series in order to solve a partial differential equation like equation \eqref{eq:richeq_1D} \parencite{fornberg_finite_2011}.  Truncating the Taylor series given by equation \eqref{eq:tayse} after the first derivative (i.e., $o = 1$) gives
\begin{align*}
  f(x) &= \frac{(x - x_0)^0}{0!}f^{(0)}(x_0) + \frac{(x - x_0)^1}{1!}f^{(1)}(x_0) + R_1 \\
       &= f(x_0) + (x - x_0)f'(x_0) + R_1
\end{align*}
Solving for $f'(x_0)$ gives
\begin{align*}
  f'(x_0) = \frac{f(x)}{x - x_0} - \frac{f(x_0)}{x - x_0} - \frac{R_1}{x - x_0} \, .
\end{align*}
Assuming that $\frac{R_1}{x - x_0}$ is sufficiently small we obtain
\begin{equation}
  \label{eq:fidifmeth}
  f'(x_0) \approx \frac{f(x) - f(x_0)}{x - x_0} \, .
\end{equation}

Equation \eqref{eq:celia_16} contains 3 terms requiring spatial discretization, namely
\begin{equation*}
  \pd{}{z} \del{ K^{n+1,m} \pd{H^{n+1,m}}{z} } \, , \, \pd{K^{n+1,m}}{z} \, , \, \text{ and } \, \pd{}{z} \del{ K^{n+1,m} \pd{\delta^m}{z} } \, .
\end{equation*}
As an example, I will explain spatial discretization of the first term above.  Spatial discretization first requires assignment of a spatial level, denoted here by $i$.  We thus obtain
\begin{equation*}
  \frac{\Delta}{\Delta z} \del{ K^{n+1,m}\frac{\Delta H^{n+1,m}}{z} }_i \, .
\end{equation*}
Expansion of the outer $\frac{\Delta}{\Delta z}$-term yields
\begin{equation*}
  \frac{1}{\Delta z} \del{ K_{i+\frac{1}{2}}^{n+1,m}\frac{\Delta H_{i+\frac{1}{2}}^{n+1,m}}{\Delta z} - K_{i-\frac{1}{2}}^{n+1,m}\frac{\Delta H_{i-\frac{1}{2}}^{n+1,m}}{\Delta z} } \, .
\end{equation*}
Expansion of the inner $\frac{\Delta}{\Delta z}$-terms yields the spatially discretized version
\begin{align*}
  &\frac{1}{\Delta z} \del{ K_{i+\frac{1}{2}}^{n+1,m}\frac{1}{\Delta z} \del{ H_{i+1}^{n+1,m} - H_{i}^{n+1,m} } - K_{i-\frac{1}{2}}^{n+1,m}\frac{1}{\Delta z} \del{ H_{i}^{n+1,m} - H_{i-1}^{n+1,m} } } \\
  = &\frac{1}{\del{ \Delta z }^2} \del{ K_{i+\frac{1}{2}}^{n+1,m} \del{ H_{i+1}^{n+1,m} - H_{i}^{n+1,m} } - K_{i-\frac{1}{2}}^{n+1,m} \del{ H_{i}^{n+1,m} - H_{i-1}^{n+1,m} } } \, .
\end{align*}

Substituting the spatially discretized versions of all 3 terms into equation \eqref{eq:celia_16} yields \parencite{celia_general_1990}
\begin{equation}
  \label{eq:celia_17}
  \begin{split}
    % \pd{}{z} \del{ K^{n+1,m} \pd{H^{n+1,m}}{z} } + \pd{K^{n+1,m}}{z} - \frac{\theta^{n+1,m} - \theta^n}{\Delta t} = \del{ \frac{1}{\Delta t} C^{n+1,m} } \delta^m - \pd{}{z} \del{ K^{n+1,m} \pd{\delta^m}{z} } \, .
    &\frac{1}{\del{ \Delta z }^2} \del{ K_{i+\frac{1}{2}}^{n+1,m} \del{ H_{i+1}^{n+1,m} - H_{i}^{n+1,m} } - K_{i-\frac{1}{2}}^{n+1,m} \del{ H_{i}^{n+1,m} - H_{i-1}^{n+1,m} } } \\
    &+ \frac{K_{i+\frac{1}{2}}^{n+1,m} - K_{i-\frac{1}{2}}^{n+1,m}}{\Delta z} - \frac{\theta_i^{n+1,m} - \theta_i^{n}}{\Delta t} \\
    = &\del{ \frac{1}{\Delta t} C^{n+1,m} } \delta^m - \frac{1}{\del{ \Delta z }^2} \del{ K_{i+\frac{1}{2}}^{n+1,m} \del{ \delta_{i+1}^{m} - \delta_{i}^{m} } - K_{i-\frac{1}{2}}^{n+1,m} \del{ \delta_{i}^{m} - \delta_{i-1}^{m} } }  \, ,
  \end{split}
\end{equation}
where $K_{i\pm\frac{1}{2}}$ is defined as the arithmetic mean between $K_i$ and $K_{i\pm1}$.  The left side of equation \eqref{eq:celia_17} is now a measure of the amount by which the $m$th iterate fails to satisfy the finite difference approximation \parencite{celia_general_1990}.

%%% Local Variables:
%%% mode: latex
%%% TeX-master: "basefile/Projekt_basefile"
%%% End:


\subsection{Matrix form}
At each time level, equation \eqref{eq:celia_17} has to be solved iteratively for each spatial level.  Additionally, 2 separate scenarios will be simulated, called ``constant pressure'', or cp,  and ``no flux'', or nf, respectively.  The scenarios differ in the treatment of the bottom node.  The cp scenario assumes a constant pressure head at the bottom node, whereas the nf scenario assumes no flux at the bottom node.  All other nodes are treated the same way in both scenarios.

To make full use of the computational capabilities of GNU Octave, equation \eqref{eq:celia_17} needs to be reformulated in matrix form as \parencite{lier_root_2006}
\begin{equation}
  \label{eq:maform}
  \mathbf{A} \cdot \boldsymbol{\delta} = \mathbf{f} \, ,
\end{equation}
where
\begin{equation}
  \label{eq:mat_A}
  \mathbf{A} = 
  \begin{bmatrix}
    \beta_1 & \gamma_1 & & & & & \\
    \alpha_2 & \beta_2 & \gamma_2 & & & & \\
    & \alpha_3 & \beta_3 & \gamma_3 & & & \\
    & & & & \alpha_{p-1} & \beta_{p-1} & \gamma_{p-1} \\
    & & & & & \alpha_p & \beta_p
  \end{bmatrix} \, ,
\end{equation}
\begin{equation}
  \label{eq:vec_delta}
  \boldsymbol{\delta} = 
  \begin{bmatrix}
    &\delta_1^m \\
    &\delta_2^m \\
    &\delta_3^m \\
    &\delta_{p-1}^m \\
    &\delta_p^m
  \end{bmatrix} \, ,
\end{equation}
and
\begin{equation}
  \label{eq:vec_f}
  \mathbf{f} =
  \begin{bmatrix}
    &f_1 \\
    &f_2 \\
    &f_3 \\
    &f_{p-1} \\
    &f_p
  \end{bmatrix} \, .
\end{equation}


The individual elements of $\mathbf A$, $\boldsymbol{\delta}$ and $\mathbf f$ are defined as follows:
\begin{enumerate}
\item for $i = 1$ (bottom spatial level):
  \begin{enumerate}
  \item cp scenario:
    \begin{equation}
      \label{eq:beta_bottom}
      \beta_i = 1
    \end{equation}
    \begin{equation}
      \label{eq:gamma_bottom}
      \gamma_i = 0
    \end{equation}
    \begin{equation}
      \label{eq:delta_bottom}
      \delta_i^m = H_i^{n+1,m+1} - H_i^{n+1,m}
    \end{equation}
    \begin{equation}
      \label{eq:f_bottom}
      f_i = 0
    \end{equation}
  \item nf scenario:
    \begin{equation}
      \label{eq:beta_bottom}
      \beta_i = \frac{C_i^{n+1,m}}{\Delta t} + \frac{K_{i+\frac{1}{2}}^{n+1,m}}{\del{ \Delta z }^2}
      % \beta_i = 1
    \end{equation}
    \begin{equation}
      \label{eq:gamma_bottom}
      \gamma_i = - \frac{K_{i+\frac{1}{2}}^{n+1,m}}{\del{ \Delta z }^2}
    \end{equation}
    \begin{equation}
      \label{eq:f_bottom}
      f_i = \frac{1}{\del{ \Delta z }^2} K_{i+\frac{1}{2}}^{n+1,m} \del{ H_{i+1}^{n+1,m} - H_{i}^{n+1,m}} + \frac{K_{i+\frac{1}{2}}^{n+1,m}}{\Delta z} - \frac{\theta_i^{n+1,m} - \theta_i^{n}}{\Delta t}
    \end{equation}
  \end{enumerate}
\item for $1 < i < p$ (intermediate spatial levels):
  \begin{equation}
    \label{eq:alpha_intermediate}
    \alpha_i = - \frac{K_{i-\frac{1}{2}}^{n+1,m}}{\del{ \Delta z }^2}
  \end{equation}
  \begin{equation}
    \label{eq:beta_intermediate}
    \beta_i = \frac{C_i^{n+1,m}}{\Delta t} + \frac{K_{i+\frac{1}{2}}^{n+1,m}}{\del{ \Delta z }^2} + \frac{K_{i-\frac{1}{2}}^{n+1,m}}{\del{ \Delta z }^2}
  \end{equation}
  \begin{equation}
    \label{eq:gamma_intermediate}
    \gamma_i = - \frac{K_{i+\frac{1}{2}}^{n+1,m}}{\del{ \Delta z }^2}
  \end{equation}
  \begin{equation}
    \label{eq:delta_intermediate}
    \delta_i^m = H_i^{n+1,m+1} - H_i^{n+1,m}
  \end{equation}
  \begin{equation}
    \label{eq:f_intermediate}
    \begin{split}
      f_i = &\frac{1}{\del{ \Delta z }^2} \del{ K_{i+\frac{1}{2}}^{n+1,m} \del{ H_{i+1}^{n+1,m} - H_{i}^{n+1,m} } - K_{i-\frac{1}{2}}^{n+1,m} \del{ H_{i}^{n+1,m} - H_{i-1}^{n+1,m} } } \\
      &+ \frac{K_{i+\frac{1}{2}}^{n+1,m} - K_{i-\frac{1}{2}}^{n+1,m}}{\Delta z} - \frac{\theta_i^{n+1,m} - \theta_i^{n}}{\Delta t}
    \end{split}
  \end{equation}
\item for $i = p$ (top spatial level):
  \begin{equation}
    \label{eq:alpha_top}
    \alpha_i = 0
  \end{equation}
  \begin{equation}
    \label{eq:beta_top}
    \beta_i = 1
  \end{equation}
  \begin{equation}
    \label{eq:delta_top}
    \delta_i^m = H_i^{n+1,m+1} - H_i^{n+1,m}
  \end{equation}
  \begin{equation}
    \label{eq:f_top}
    f_i = 0
  \end{equation}
\end{enumerate}
The model rests on the assumption that appropriate constitutive relationshiphs between $\theta$ and $h$ and between $K$ and $\theta$ exist.  These relationships are modelled with the help of the equations laid out by \textcite{genuchten_closed-form_1980}, namely
\begin{equation}
  \label{eq:vange_K}
  % K(h) = K_{sat} \frac{\del{ 1 - \del{ \omega \abs{h} }^{v-1} \del{ 1 + \del{ \omega \abs{h} }^v }^{-u} }^2}{\del{ 1 + \del{ \omega \abs{h} }^v }^{\frac{u}{2}}} \, ,
  K(\Theta) = K_{sat} \Theta^{\lambda} \del{ 1 - \del{ 1 - \Theta^{\frac{1}{u}} }^u }^2
\end{equation}
\begin{equation}
  \label{eq:vange_theta}
  \theta(h) = \theta_r + \frac{\theta_s - \theta_r}{\del{ 1 + \del{ \omega \abs{h} }^v }^u} \, ,
\end{equation}
and
\begin{equation}
  \label{eq:vange_C}
  C(h) = \frac{-\omega u \del{ \theta_s - \theta_r }}{1 - u} \Theta^{\frac{1}{u}} \del{ 1 - \Theta^{\frac{1}{u}} }^u \, ,
\end{equation}
where
\begin{equation*}
  u = 1 - \frac{1}{v} \, ,
\end{equation*}
\begin{equation*}
  % \Theta = \del{ \frac{1}{1 + \del{ \omega \abs{h} }^v} }^u \, ,
  \Theta = \frac{\theta - \theta_r}{\theta_s - \theta_r} \, ,
\end{equation*}
and where $\theta_r$, $\theta_s$, $\lambda$, $\omega$, $K_{sat}$, and $u$ are parameters that need to be determined for a given isotropic soil.

Taking equations \eqref{eq:vange_K}, \eqref{eq:vange_theta}, and \eqref{eq:vange_C} into account and provided appropriate initial and boundary values are supplied, $\mathbf{A}$ and $\mathbf{f}$ in equation \eqref{eq:maform} can be calculated for a given iterative step.  In turn, vector $\delta$ can be computed by left division as follows:
\begin{equation}
  \label{eq:maform_ledi}
  \boldsymbol{\delta} = \mathbf{A \backslash f}
\end{equation}
This equation is solved in an iterative manner for every time level.  At the end of each iterative step the algorithm checks whether the maximum of $\boldsymbol{\delta}$ exceeds an arbitrary threshold $\delta_{th}$.
If the threshold is exceeded, it means the estimated value of $H$ differed too much from the estimate of the previous iterative step and the iteration needs to continue.
% First the values of both $H_1$ and $H_P$ are imposed (cp scenario) or only the value of $H_p$ is imposed (nf scenario).  Then the sum of $\delta_i^m$ and $H_i^{m+1,n}$ is used as the estimate of $H_i$ for the next iterative step.  
If the threshold is not exceeded, it means the estimated value of $H$ is considered acceptable and the iteration can be exited.
% First the sum of $\delta_i^m$ and $H_i^{m+1,n}$ is used as the value of $H_i$ for the next time step.  Then the values of both $H_1$ and $H_P$ are imposed (cp scenario) or only the value of $H_p$ is imposed (nf scenario).  Afterwards the iterative loop is excited and calculations for the new time step begin.

%%% Local Variables:
%%% mode: latex
%%% TeX-master: "basefile/Projekt_basefile"
%%% End:


% There exist 3 standard forms of the Richards equation \parencite{celia_general_1990}:  \\
% $h$ based form
% \begin{equation}
  % \label{eq:h_based_richeq}
  % C(h) \frac{\partial h}{\partial t} - \nabla \cdot K(h)\nabla h - \frac{\partial K(h)}{\partial z} = 0
% \end{equation}
% $\theta$ based form
% \begin{equation}
  % \label{eq:theta_base_richeq}
  % \frac{\partial \theta}{\partial t} - \nabla \cdot D(\theta)\nabla \theta - \frac{\partial K(h)}{\partial z} = 0
% \end{equation}
% mixed form
% \begin{equation}
  % \label{eq:mixed_richeq}
  % \frac{\partial \theta}{\partial t} - \nabla \cdot K(h)\nabla h - \frac{\partial K(h)}{\partial z} = 0
% \end{equation}
% The model employs the mixed

% The hydraulic conductivity for a given pressure head is estimated using the equation given by \textcite{genuchten_closed-form_1980}:
% \begin{equation}
  % \label{eq:vange_hycon}
  % K(h) = K_s 
% \end{equation}

%%% Local Variables:
%%% mode: latex
%%% TeX-master: "basefile/Projekt_basefile"
%%% End:
