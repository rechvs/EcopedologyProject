In the model described here, water movement is explained with the help of the Richards equation \parencite{richards_capillary_1931}.  The Richards equation is the result of combining Darcy’s law with the continuity principle \parencite{hillel_environmental_1998}.  According to Darcy’s law, water flux $q$ through a given (isotropic) volume of soil is proportional to the hydraulic head drop $\nabla \del{ h + z }$ between inflow and outflow boundary, with hydraulic conductivity $K$ acting as the proportionality factor linking both terms:
\begin{equation}
  \label{eq:dareq}
  q = K(h) \nabla \del{ h + z }
\end{equation}
The continuity principle states that the temporal change in water content of a given volume of soil $\frac{\partial \theta}{\partial t}$ must be equal to the spatial change in water flux $\nabla \cdot q$ from/to this volume:
\begin{equation}
  \label{eq:contprinc}
  \pd{\theta}{t}  = \nabla \cdot q
\end{equation}
Replacing $q$ in equation (\ref{eq:contprinc}) with the right hand side of equation (\ref{eq:dareq}) gives the Richard equation:
\begin{equation}
  \label{eq:richeq}
  \pd{\theta}{t} = \nabla \cdot \del{ K(h) \nabla \del{ h + z } }
\end{equation}
Equation (\ref{eq:richeq}) can be further simplified by applying the vector differential operator $\nabla$: % cp. Notes on Celia et al.1990, p. 24 ff.
\begin{align*}
  \pd{\theta}{t} &= \nabla \cdot \del{ K(h) \nabla \del{ h + z } } \\
                                     &= \nabla \cdot\del{ K(h) \del{ \nabla h + \nabla z } } \\
                                     &= \nabla \cdot\del{ K(h) \del{ \nabla h + \del{ \pd{z}{x},\pd{z}{y},\pd{z}{z} } } } \\
                                     &= \nabla \cdot\del{ K(h) \del{ \nabla h + \del{ 0,0,1 } } } \\
                                     &= \nabla \cdot \del{ K(h) \nabla h + K(h) \del{ 0,0,1 } } \\
                                     &= \nabla \cdot \del{ K(h) \nabla h + \del{ 0,0,K(h) } } \\
                                     &= \nabla \cdot \del{ K(h) \nabla h } + \nabla \cdot \del{ 0,0,K(h) } \\
                                     &= \nabla \cdot \del{ K(h) \nabla h } + \del{ \pd{0}{x},\pd{0}{y},\pd{K(h)}{z}} \\
                                     &= \nabla \cdot \del{ K(h) \nabla h } + \pd{K(h)}{z}
\end{align*}
The Richards equation is thus
\begin{equation}
  \label{eq:richeq_3D}
  \pd{\theta}{t} = \nabla \cdot \del{ K(h) \nabla h } + \pd{K(h)}{z} \, .
\end{equation}
The model presented here is a one-dimensional model, considering only the vertical spatial dimension and disregarding $x$ and $y$ directions.  Therefore, equation \eqref{eq:richeq_3D} needs to be adjusted to the one-dimensional case by setting $\pd{f(x,y,z)}{x} = 0$ and $\pd{f(x,y,z)}{y} = 0$:
\begin{align*}
  \pd{\theta}{t} &= \nabla \cdot \del{ K(h) \nabla h } + \pd{K(h)}{z} \\
                 &= \nabla \cdot \del{ K(h) \del{ \pd{h}{x},\pd{h}{y},\pd{h}{z} } } + \pd{K(h)}{z} \\
                 % &= \nabla \cdot \del{ K(h) \del{ 0, 0, \pd{h}{z} } } + \pd{K(h)}{z} \\
                 % &= \nabla \cdot  \del{ 0, 0, K(h) \pd{h}{z} } + \pd{K(h)}{z} \\
                 &= \nabla \cdot \del{ K(h) \pd{h}{x},K(h) \pd{h}{y},K(h) \pd{h}{z} } + \pd{K(h)}{z} \\
                 &= \pd{}{x} \del{ K(h) \pd{h}{x} } + \pd{}{y} \del{ K(h) \pd{h}{y} } + \pd{}{z} \del{ K(h) \pd{h}{z} } + \pd{K(h)}{z} \\
                 &= 0 + 0 + \pd{}{z} \del{ K(h) \pd{h}{z} } + \pd{K(h)}{z} \, .
\end{align*}
The Richards equation for the one-dimensional case considering only the vertical spatial dimension is thus
\begin{equation}
  \label{eq:richeq_1D}
  % \pd{\theta}{t} - \nabla \cdot \del{ K(h) \nabla h } - \pd{K(h)}{z} = 0 \, .
  \pd{\theta}{t} = \frac{\partial}{\partial z} \del{ K(h) \pd{h}{z} } + \pd{K(h)}{z} \, .
\end{equation}
% Equation (\ref{eq:richeq_pde}) is also called the ``mixed form'' of the Richards equation, since it is a mixture of the ``$h$-based'' and the ``$\theta$-based'' form \parencite{celia_general_1990}.
The relationships between $\theta$ and $h$ on the one hand, and $K$ and $h$ on the other hand are nonlinear \parencite{celia_general_1990}.  Therefore, numerical approximation is a widely used approach for solving equation \eqref{eq:richeq_1D}, requiring both temporal discretization (i.e., discretization with respect to $t$) and spatial discretization (i.e., discretization with respect to $z$) of the equation.

In general, 2 approaches for temporally discretizing differential equations exist:  explicit methods and implicit methods.  Explicit methods estimate the value of a function at time $t_{n+1} = t_n + \Delta t$ based on the value at time $t_n$. For example, the Euler forward method uses the equation
\begin{equation}
  \label{eq:euformeth}
  s_{n+1} = s_n + (t_{n+1} - t_n) f(t_n,s_n)
\end{equation}
to obtain the value $s_{n+1}$.  Since the method attempts to calculate the value at $t_{n+1}$ based on information valid at $t_n$ it is prone to error and instable results \parencite{Weisstein2017a}.  Implicit methods do not suffer from these shortcomings.  The model presented here employs the Euler backward method, which is an example of an implicit method \parencite{Weisstein2017b}.

\subsection{Euler backward method}
The Euler backward method is based on the mean value theorem.  For the mean value theorem to be applicable, a function $f(x)$ must be differentiable in the open interval \intoo{t_0,t_1} and continuous in the closed interval \intcc{t_0,t_1} \parencite{Weisstein2017c}.  If these conditions are met, then there is at least one point $t_2$ in \intoo{t_0,t_1} such that
\begin{equation*}
  % f'(t_2) = \frac{f(t_1) - f(t_0)}{t_1 - t_0} \, .
% \end{equation*}
% Rearranging gives
% \begin{equation*}
  f(t_1) = f(t_0) + (t_1 - t_0) f'(t_2) \, .
\end{equation*}

% Let us assume we are given the ordinary differential equation
% \begin{equation*}
  % y'(t) = f(t, y(t)) \, .
% \end{equation*}
% Applying the mean value theorem we obtain
% \begin{equation*}
  % y(t_1) = y(t_0) + (t_1 - t_0) y'(t_2) \, .
% \end{equation*}
The Euler backward method works by assuming that $t_2 = t_1$ \parencite{hairer_solving_2009}, so that
\begin{equation*}
  % y(t_1) = y(t_0) + (t_1 - t_0) y'(t_1) \, .
  f(t_1) = f(t_0) + (t_1 - t_0) f'(t_1) \, .
\end{equation*}
Applying the Euler backward method coupled with the Picard iteration, we obtain the temporally discretized version of equation \eqref{eq:richeq_1D} \parencite{celia_general_1990}, 
\begin{equation}
  \label{eq:celia_14}
  \frac{\theta^{n+1,m+1} - \theta^n}{\Delta t} - \frac{\partial}{\partial z} \del{ K^{n+1,m}\pd{H^{n+1,m+1}}{z} } - \pd{K^{n+1,m}}{z} = 0 \, ,
\end{equation}
where $n$ and $m$ denote time level and iteration level, respectively.  The value $\theta^{n+1,m+1}$ in equation \eqref{eq:celia_14} can be approximated using a truncated Taylor series, which will be explained in the following section.


%%% Local Variables:
%%% mode: latex
%%% TeX-master: "basefile/Projekt_basefile"
%%% End:


\subsection{Taylor series}
A Taylor series is a series expansion of a function about a point \parencite{Weisstein2017}.
% A one-dimensional Taylor series expansion of a real function $f\del{ x }$ about a point $x = a$ is given by
% \begin{equation*}
  % f\del{ x }=f\del{ a }+f'\del{ a }\del{ x-a }+\frac{f''\del{ a }}{2!}\del{ x-a }^2+\frac{f'''\del{ a }}{2!}\del{ x-a }^3+\mathellipsis+\frac{f^{\del{ o }}\del{ a }}{n!}\del{ x-a }^n+\mathellipsis
% \end{equation*}
The Taylor series of a function $f\del{ x }$ can be derived by first noting that the integral of the $\del{ o+1 }$st derivative $f^{\del{ o+1 }}$ of $f\del{ x }$ from the point $x_0$ to an arbitrary point $x = x_0 + \Delta x$ is given by
\begin{align*}
  \int_{x_0}^x{f^{\del{ o+1 }}\del{ x }}\dif x &= \sbr{ f^{\del{ o }}\del{ x } }_{x_0}^x \\
                               &= f^{\del{ o }}\del{ x } - f^{\del{ o }}\del{ x_0 } \, .
\end{align*}
A second integration gives
\begin{align*}
  \int_{x_0}^x\int_{x_0}^x{f^{\del{ o+1 }}\del{ x }} \del{ \dif x }^2 &= \int_{x_0}^x{\del{ f^{\del{ o }}\del{ x } - f^{\del{ o }}\del{ x_0 } }}\dif x \\
                                    &= \int_{x_0}^x{f^{\del{ o }}\del{ x }}\dif x - \int_{x_0}^x{f^{\del{ o }}\del{ x_0 }}\dif x \\
                                    &= \sbr{ f^{\del{ o-1 }}\del{ x } }_{x_0}^x - \sbr{ x f^{\del{ o }}\del{ x_0 } }_{x_0}^x \\
                                    &= f^{\del{ o-1 }}\del{ x } - f^{\del{ o-1 }}\del{ x_0 } - \del{ x f^{\del{ o }}\del{ x } - x_0 f^{\del{ o }}\del{ x_0 } } \\
                                    &= f^{\del{ o-1 }}\del{ x } - f^{\del{ o-1 }}\del{ x_0 } - \del{ x - x_0 } f^{\del{ o }}\del{ x_0 } \, .
\end{align*}
A third integration gives
\begin{align*}
  &\int_{x_0}^x\int_{x_0}^x\int_{x_0}^x{f^{\del{ o+1 }}\del{ x }} \del{ \dif x }^3 \\
  = &\int_{x_0}^x{\del{ f^{\del{ o-1 }}\del{ x } - f^{\del{ o-1 }}\del{ x_0 } - \del{ x - x_0 } f^{\del{ o }}\del{ x_0 } }}\dif x \\
  = &\int_{x_0}^x{f^{\del{ o-1 }}\del{ x }}\dif x - \int_{x_0}^x{f^{\del{ o-1 }}\del{ x_0 }}\dif x - \int_{x_0}^x\del{ x - x_0 }{f^{\del{ o }}\del{ x_0 }}\dif x \\
  = &\int_{x_0}^x{f^{\del{ o-1 }}\del{ x }}\dif x - \int_{x_0}^x{f^{\del{ o-1 }}\del{ x_0 }}\dif x - \int_{x_0}^x\del{x f^{\del{ o }}\del{ x_0 } - x_0 f^{\del{ o }}\del{ x_0 }}\dif x \\
  = &\int_{x_0}^x{f^{\del{ o-1 }}\del{ x }}\dif x - \int_{x_0}^x{f^{\del{ o-1 }}\del{ x_0 }}\dif x - \del{ \int_{x_0}^x{x f^{\del{ o }}\del{ x_0 }}\dif x - \int_{x_0}^x{x_0 f^{\del{ o }}\del{ x_0 }}\dif x } \\
  = &\sbr{ f^{\del{ o-2 }}\del{ x } }_{x_0}^x -\sbr{ x f^{\del{ o-1 }}\del{ x_0 } }_{x_0}^x - \del{ \sbr{ \frac{1}{2}x^2f^{\del{ o }}\del{ x_0 } }_{x_0}^x - \sbr{ xx_0f^{\del{ o }}\del{ x_0 } }_{x_0}^x } \\
  = &f^{\del{ o-2 }}\del{ x }
    - f^{\del{ o-2 }}\del{ x_0 }
    - \del{ x f^{\del{ o-1 }}\del{ x_0 }
    - x_0 f^{\del{ o-1 }}\del{ x_0 } } \\
  &- \del{ \del{ \frac{1}{2} x^2 f^{\del{ o }}\del{ x_0 }
    - \frac{1}{2} x_0^2 f^{\del{ o }}\del{ x_0 } }
    - \del{ x x_0 f^{\del{ o }}\del{ x_0 }
    - x_0 x_0 f^{\del{ o }}\del{ x_0 } } } \\
  = &f^{\del{ o-2 }}\del{ x }
    - f^{\del{ o-2 }}\del{ x_0 }
    - \del{ x - x_0 } f^{\del{ o-1 }}\del{ x_0 } - \del{ \frac{1}{2} x^2 - \frac{1}{2} x_0^2 - x x_0 + x_0^2 } f^{\del{ o }}\del{ x_0 } \\
  = &f^{\del{ o-2 }}\del{ x }
    - f^{\del{ o-2 }}\del{ x_0 }
    - \del{ x - x_0 } f^{\del{ o-1 }}\del{ x_0 } - \del{ \frac{1}{2} x^2 - x x_0 + \frac{1}{2} x_0^2 } f^{\del{ o }}\del{ x_0 } \\
  = &f^{\del{ o-2 }}\del{ x }
    - f^{\del{ o-2 }}\del{ x_0 }
    - \del{ x - x_0 } f^{\del{ o-1 }}\del{ x_0 } - \frac{1}{2} \del{ x^2 - 2 x x_0 + x_0^2 } f^{\del{ o }}\del{ x_0 } \\
  = &f^{\del{ o-2 }}\del{ x }
    - f^{\del{ o-2 }}\del{ x_0 }
    - \del{ x - x_0 } f^{\del{ o-1 }}\del{ x_0 } - \frac{1}{2} \del{ x - x_0 }^2 f^{\del{ o }}\del{ x_0 } \\
  = &f^{\del{ o-2 }}\del{ x }
    - f^{\del{ o-2 }}\del{ x_0 }
    - \del{ x - x_0 } f^{\del{ o-1 }}\del{ x_0 } - \frac{\del{ x - x_0 }^2}{2 !}  f^{\del{ o }}\del{ x_0 } \, .
\end{align*}
Continuing up to $o + 1$ integrations gives
\begin{align*}
  &\underbrace{\int \hspace{-4mm} \hdots \hspace{-1mm} \int_{x_0}^x}_{o+1}{f^{\del{ o+1 }}\del{ x }\del{ \dif x }^{\del{ o+1 }}} \\
  = &f\del{ x } - f\del{ x_0 } - \del{ x - x_0 } f'\del{ x_0 } - \frac{\del{ x - x_0 }^2}{2!} f''\del{ x_0 } - \frac{\del{ x - x_0 }^3}{3!} f'''\del{ x_0 } \\
  &- \mathellipsis - \frac{\del{ x - x_0 }^o}{o!} f^{\del{ o }}\del{ x_0 } \, .
\end{align*}
Solving for $f\del{ x }$ gives
\begin{align*}
  f\del{ x } = &f\del{ x_0 }
           + \del{ x - x_0 } f'\del{ x_0 }
           + \frac{\del{ x-x_0 }^2}{2!} f''\del{ x_0 } \\
         &+ \frac{\del{ x - x_0 }^3}{3!} f'''\del{ x_0 }
           + \mathellipsis \frac{\del{ x - x_0 }^o}{o!} f^{\del{ o }}\del{ x_0 }
           + \underbrace{\int \hspace{-4mm} \hdots \hspace{-1mm} \int_{x_0}^x}_{o+1}{f^{\del{ o+1 }}\del{ x }\del{ \dif x }^{\del{ o+1 }}} \\
  = &\sum_{k=0}^o{\frac{\del{ x - x_0 }^k}{k!}}f^{\del{ k }}\del{ x_0 } + \underbrace{\int \hspace{-4mm} \hdots \hspace{-1mm} \int_{x_0}^x}_{o+1}{f^{\del{ o+1 }}\del{ x }\del{ \dif x }^{\del{ o+1 }}} \, .
\end{align*}
The last term on the right hand side of the equation above is known as the Lagrange remainder
\begin{align*}
  R_o = &\underbrace{\int \hspace{-4mm} \hdots \hspace{-1mm} \int_{x_0}^x}_{o+1}{f^{\del{ o+1 }}\del{ x }\del{ \dif x }^{\del{ o+1 }}} \, .
\end{align*}
The one-dimensional Taylor series of function $f\del{ x }$ is thus
\begin{equation}
  \label{eq:tayse}
  f\del{ x } = \sum_{k=0}^o{\frac{\del{ x - x_0 }^k}{k!}}f^{\del{ k }}\del{ x_0 } + R_o \, .
\end{equation}
The value $\theta^{n+1,m+1}$ in equation \eqref{eq:celia_14} can be estimated using equation \eqref{eq:tayse} by setting $ f\del{ x } = \theta^{n+1,m+1}, x = H^{n+1,m+1}, o = 1, x_0 = H^{n+1,m}, f\del{ x_0 } = \theta^{n+1,m}, \text{ and } f'\del{ x_0 } = \evalat{\od{\theta}{h}}{n+1,m}$ \parencite{celia_general_1990}:
\begin{align*}
  % \theta^{n+1,m+1} = \sum_{k=0}^1\frac{\del{ H^{n+1,m+1} - H^{n+1,m} }^k}{k!}f^{\del{ k }}\del{ H^{n+1,m} } + R_1
  \theta^{n+1,m+1} &= \sum_{k=0}^1\frac{\del{ H^{n+1,m+1} - H^{n+1,m} }^k}{k!}f^{\del{ k }}\del{ H^{n+1,m} } + R_1 \\
                   &= \frac{\del{ H^{n+1,m+1} - H^{n+1,m} }^0}{0!} f^{\del{ 0 }}\del{ H^{n+1,m} } + \frac{\del{ H^{n+1,m+1} - H^{n+1,m} }^1}{1!} f^{\del{ 1 }}\del{ H^{n+1,m} } + R_1 \\
                   &= f\del{ H^{n+1,m} } + \del{ H^{n+1,m+1} - H^{n+1,m} } f'\del{ H^{n+1,m} } + 0 \del{ \delta^2 } \\
  \label{eq:celia_15}
                   &= \theta^{n+1,m} + \evalat{\od{\theta}{h}}{n+1,m} \del{ H^{n+1,m+1} - H^{n+1,m} } + 0 \del{ \delta^2 }
\end{align*}
Neglecting all terms higher than linear in the equation above we obtain
\begin{align}
  \theta^{n+1,m+1} &= \theta^{n+1,m} + \evalat{\od{\theta}{h}}{n+1,m} \del{ H^{n+1,m+1} - H^{n+1,m} } \\
  \label{eq:celia_15.25}
                   &= \theta^{n+1,m} + C^{n+1,m} \delta^m
\end{align}
Substituting equation \eqref{eq:celia_15.25} into equation \eqref{eq:celia_14} we obtain
\begin{equation*}
  0 = \frac{C^{n+1,m}}{\Delta t} \delta^m + \frac{\theta^{n+1,m}  - \theta^n}{\Delta t} - \pd{}{z} \del{ K^{n+1,m}\pd{H^{n+1,m+1}}{z} } - \pd{K^{n+1,m}}{z}
\end{equation*}
This equation can be rewritten in terms of the increment in iteration $\delta ^m$ (see page \pageref{celia_15.5_transformation}):
\newpage
\label{celia_15.5_transformation}
\begin{landscape}
  \begin{align*}
    0 &= \frac{C^{n+1,m}}{\Delta t} \delta^m + \frac{\theta^{n+1,m} - \theta^n}{\Delta t} - \pd{}{z} \del{ K^{n+1,m}\pd{H^{n+1,m+1}}{z} } - \pd{K^{n+1,m}}{z} \quad \Bigr\rvert - \frac{\theta^{n+1,m} - \theta^n}{\Delta t} \\
    - \frac{\theta^{n+1,m} - \theta^n}{\Delta t} &= \frac{C^{n+1,m}}{\Delta t} \delta^m - \pd{}{z} \del{ K^{n+1,m}\pd{H^{n+1,m+1}}{z} } - \pd{K^{n+1,m}}{z} \quad \Bigr\rvert + \pd{K^{n+1,m}}{z} \\
    \pd{K^{n+1,m}}{z} - \frac{\theta^{n+1,m} - \theta^n}{\Delta t} &= \frac{C^{n+1,m}}{\Delta t} \delta^m - \pd{}{z} \del{ K^{n+1,m}\pd{H^{n+1,m+1}}{z} } \quad \Bigr\rvert - \pd{}{z} \del{ K^{n+1,m}\pd{}{z}\del{ H^{n+1,m} - H^{n+1,m} } } \\
    \pd{K^{n+1,m}}{z} - \frac{\theta^{n+1,m} - \theta^n}{\Delta t} - \pd{}{z} \del{ K^{n+1,m}\pd{}{z}\del{ H^{n+1,m} - H^{n+1,m} } } &= \frac{C^{n+1,m}}{\Delta t} \delta^m - \pd{}{z} \del{ K^{n+1,m}\pd{H^{n+1,m+1}}{z} } - \pd{}{z} \del{ K^{n+1,m}\pd{}{z}\del{ H^{n+1,m} - H^{n+1,m} } } \\
    \pd{K^{n+1,m}}{z} - \frac{\theta^{n+1,m} - \theta^n}{\Delta t} - 0 &= \frac{C^{n+1,m}}{\Delta t} \delta^m - \pd{}{z} \del{ K^{n+1,m}\pd{H^{n+1,m+1}}{z} } - \pd{}{z} \del{ K^{n+1,m}\del{ \pd{H^{n+1,m}}{z} - \pd{H^{n+1,m}}{z} } } \\
    \pd{K^{n+1,m}}{z} - \frac{\theta^{n+1,m} - \theta^n}{\Delta t} &= \frac{C^{n+1,m}}{\Delta t} \delta^m - \pd{}{z} \del{ K^{n+1,m}\pd{H^{n+1,m+1}}{z} - K^{n+1,m}\del{ \pd{H^{n+1,m}}{z} - \pd{H^{n+1,m}}{z} } } \\
      &= \frac{C^{n+1,m}}{\Delta t} \delta^m - \pd{}{z} \del{ K^{n+1,m} \del{ \pd{H^{n+1,m+1}}{z} - \pd{H^{n+1,m}}{z} + \pd{H^{n+1,m}}{z} } } \\
      &= \frac{C^{n+1,m}}{\Delta t} \delta^m - \pd{}{z} \del{ K^{n+1,m} \pd{}{z} \del{ H^{n+1,m+1} - H^{n+1,m} } + K^{n+1,m} \pd{H^{n+1,m}}{z} } \\
      &= \frac{C^{n+1,m}}{\Delta t} \delta^m - \pd{}{z} \del{ K^{n+1,m} \pd{\delta^m}{z} } - \pd{}{z} \del{ K^{n+1,m} \pd{H^{n+1,m}}{z} } \quad \Bigr\rvert + \pd{}{z} \del{ K^{n+1,m} \pd{H^{n+1,m}}{z} } \\
    \pd{}{z} \del{ K^{n+1,m} \pd{H^{n+1,m}}{z} } + \pd{K^{n+1,m}}{z} - \frac{\theta^{n+1,m} - \theta^n}{\Delta t} &= \frac{C^{n+1,m}}{\Delta t} \delta^m - \pd{}{z} \del{ K^{n+1,m} \pd{\delta^m}{z} }
  \end{align*}
\end{landscape}
\newpage
By switching left hand side and right hand side in the last equation above we obtain \parencite{celia_general_1990}
\begin{equation}
  \label{eq:celia_16}
  \frac{C^{n+1,m}}{\Delta t} \delta^m - \pd{}{z} \del{ K^{n+1,m} \pd{\delta^m}{z} } = \pd{}{z} \del{ K^{n+1,m} \pd{H^{n+1,m}}{z} } + \pd{K^{n+1,m}}{z} - \frac{\theta^{n+1,m} - \theta^n}{\Delta t} \, .
\end{equation}

%%% Local Variables:
%%% mode: latex
%%% TeX-master: "basefile/Projekt_basefile"
%%% End:


% Discretization of the spatial domain can be achieved by implementing the finite difference method, which itself is based on the Taylor series.

% \subsection{Finite difference method}
% The finite difference method employs a Taylor series in order to solve a partial differential equation like equation \eqref{eq:richeq_pde} 


%%% Local Variables:
%%% mode: latex
%%% TeX-master: "basefile/Projekt_basefile"
%%% End:



% There exist 3 standard forms of the Richards equation \parencite{celia_general_1990}:  \\
% $h$ based form
% \begin{equation}
  % \label{eq:h_based_richeq}
  % C(h) \frac{\partial h}{\partial t} - \nabla \cdot K(h)\nabla h - \frac{\partial K(h)}{\partial z} = 0
% \end{equation}
% $\theta$ based form
% \begin{equation}
  % \label{eq:theta_base_richeq}
  % \frac{\partial \theta}{\partial t} - \nabla \cdot D(\theta)\nabla \theta - \frac{\partial K(h)}{\partial z} = 0
% \end{equation}
% mixed form
% \begin{equation}
  % \label{eq:mixed_richeq}
  % \frac{\partial \theta}{\partial t} - \nabla \cdot K(h)\nabla h - \frac{\partial K(h)}{\partial z} = 0
% \end{equation}
% The model employs the mixed

% The hydraulic conductivity for a given pressure head is estimated using the equation given by \textcite{genuchten_closed-form_1980}:
% \begin{equation}
  % \label{eq:vange_hycon}
  % K(h) = K_s 
% \end{equation}

%%% Local Variables:
%%% mode: latex
%%% TeX-master: "basefile/Projekt_basefile"
%%% End:
