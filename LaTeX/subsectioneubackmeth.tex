The Euler backward method is based on the mean value theorem.  For the mean value theorem to be applicable, a function $f(x)$ must be differentiable in the open interval \intoo{t_0,t_1} and continuous in the closed interval \intcc{t_0,t_1} \parencite{Weisstein2017c}.  If these conditions are met, then there is at least one point $t_2$ in \intoo{t_0,t_1} such that
\begin{equation*}
  % f'(t_2) = \frac{f(t_1) - f(t_0)}{t_1 - t_0} \, .
% \end{equation*}
% Rearranging gives
% \begin{equation*}
  f(t_1) = f(t_0) + (t_1 - t_0) f'(t_2) \, .
\end{equation*}

Let us assume we are given the ordinary differential equation
\begin{equation*}
  y'(t) = f(t, y(t)) \, .
\end{equation*}
Applying the mean value theorem we obtain
\begin{equation*}
  y(t_1) = y(t_0) + (t_1 - t_0) y'(t_2) \, .
\end{equation*}
The Euler backward method works by assuming that $t_2 = t_1$ \parencite{hairer_solving_2009}, so that
\begin{equation*}
  y(t_1) = y(t_0) + (t_1 - t_0) y'(t_1) \, .
\end{equation*}
Applying the Euler backward method coupled with the Picard iteration to equation \eqref{eq:richeq_1D}, we obtain \parencite{celia_general_1990}
\begin{equation}
  \label{eq:celia_14}
  \frac{\theta^{n+1,m+1} - \theta^n}{\Delta t} - \frac{\partial}{\partial z} \del{ K^{n+1,m}\pd{H^{n+1,m+1}}{z} } - \pd{K^{n+1,m}}{z} = 0 \, ,
\end{equation}
where $n$ and $m$ denote time level and iteration level, respectively.  The value $\theta^{n+1,m+1}$ can be approximated using a truncated Taylor series, which I will explain in the following section.


%%% Local Variables:
%%% mode: latex
%%% TeX-master: "basefile/Projekt_basefile"
%%% End:
