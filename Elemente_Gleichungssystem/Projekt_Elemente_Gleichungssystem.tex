\documentclass[a4paper,12pt] {article}
\usepackage[ngerman]{babel}
\usepackage[utf8]{inputenc}
\usepackage{graphicx}
\usepackage{url}
\usepackage{float}
\usepackage{amsmath}
\usepackage{fontspec}
\usepackage{unicode-math}
\usepackage{textcomp}
\usepackage[font={small},labelfont={bf},singlelinecheck={false},justification=centering]{caption}
\usepackage[a4paper,margin=2.5cm]{geometry}
%\usepackage[a4paper,inner=2.5cm,outer=3.5cm]{geometry} % bei Erstellung von zweiseitig zu druckenden Dokumenten (\documentclass[twoside... muss gesetzt werden);
\usepackage{booktabs}
\usepackage{tabularx}
\usepackage{longtable}
%\usepackage{fixltx2e} % verträgt sich nicht mit der [H]-Option von float-Umgebungen
\usepackage{amsthm}
%\usepackage{floatrow} % verträgt sich nicht mit float
\usepackage{enumitem}
\usepackage{calc}
\usepackage[colorlinks=false]{hyperref}
\usepackage{setspace}
\usepackage{siunitx} % notwendig für Einheiten in textmode s. http://tex.stackexchange.com/questions/9043/should-greek-letters-inserted-in-text-using-math-mode-mostly-always-be-italic
\usepackage{xcolor}

% Bibliographie. %%%%%%%%%%%%%%%%%%%%%%%%%%%%%%%%%%%%%%%%%%%%%%%%%%%%%%%%%%%%%%%%%%%%%%%%%%%%%%%%%%%%%%%%%%%%%%%%%%%%%%%%%%%%%%%%%%%%%%%%%%%%%%%%%%%%%%%%%%%%

\usepackage[backend=biber,
bibstyle=bibstyle_HA,
citestyle=citestyle_HA,
natbib=false,
mcite=false,
sorting=nyt,
sortcase=false,
sortcites=true,
maxbibnames=99,
autocite=plain,
language=autobib,
hyperref=false,
urldate=long,
dateabbrev=false,
firstinits=true,
bibencoding=utf8 % s.S. 42 ff. biblatex-Dokumentation
]{biblatex} % s.S. 45 ff. biblatex-Dokumentation

\addbibresource[location=local]{../../Literatur/Projekt_Literaturliste_abstracts.bib} % s. S. 71 f. biblatex-Dokumentation

\renewcommand*{\nameyeardelim}{\space} % http://tex.stackexchange.com/questions/134063/how-to-add-a-comma-between-author-and-year
\renewcommand*{\revsdnamepunct}{\space}
\renewcommand*{\finentrypunct}{}
\renewcommand*\finalnamedelim{%
  \ifnumgreater{\value{liststop}}{2}{\addspace\bibstring{and}\addspace}%
  {\addspace\biband\addspace}}% http://tex.stackexchange.com/questions/67621/biblatex-have-and-in-the-citation-but-in-the-bibliography

\DeclareNameAlias{sortname}{last-first} % http://tex.stackexchange.com/questions/12806/guidelines-for-customizing-biblatex-styles/13076#13076

\DeclareNosort{} % http://tex.stackexchange.com/questions/171492/biblatex-biber-is-incorrectly-sorting-entries-with-hyphens-in-their-respective-a
\DeclareNoinit{
  \noinit{\regexp{[\x{2bf}\x{2018}]}}} % http://tex.stackexchange.com/questions/171492/biblatex-biber-is-incorrectly-sorting-entries-with-hyphens-in-their-respective-a

\newcommand{\biband}{\ifcurrentname{labelname}{\addspace \& \addspace}{\addspace \addcomma \addspace}} % http://tex.stackexchange.com/questions/67621/biblatex-have-and-in-the-citation-but-in-the-bibliography

% Längen, Abstände etc. (nicht via Einstellungen zu Paketen) %%%%%%%%%%%%%%%%%%%%%%%%%%%%%%%%%%%%%%%%%%%%%%%%%%%%%%%%%%%%%%%%%%%%%%%%%%%%%%%%%%%%%%%%%%%%%%%%

\setlength{\parindent}{0 cm} % keine Einrückung
\setlength{\parskip}{10 pt} % „Nach“ (s. S. 2 Formatvorlage)
\setlength{\abovecaptionskip}{0.5\baselineskip}
\setlength{\belowcaptionskip}{0.5\baselineskip}
\onehalfspacing % \usepackage{setspace} % Zeilenabstand

% Eigene Befehle und Umgebungen. %%%%%%%%%%%%%%%%%%%%%%%%%%%%%%%%%%%%%%%%%%%%%%%%%%%%%%%%%%%%%%%%%%%%%%%%%%%%%%%%%%%%%%%%%%%%%%%%%%%%%%%%%%%%%%%%%%%%%%%%%%%%

\newcolumntype{C}{>{\centering\arraybackslash}X}
\newcolumntype{L}{>{\raggedright\arraybackslash}X}

% Einstellungen zu Paketen. %%%%%%%%%%%%%%%%%%%%%%%%%%%%%%%%%%%%%%%%%%%%%%%%%%%%%%%%%%%%%%%%%%%%%%%%%%%%%%%%%%%%%%%%%%%%%%%%%%%%%%%%%%%%%%%%%%%%%%%%%%%%%%%%%

% enumitem:

\setlist{
  topsep=0pt,
  partopsep=0pt,
  parsep=0.25\baselineskip
}

% fontspec:

\setmainfont{TeX Gyre Termes}

% unicode-math:

\setmathfont[math-style=ISO,bold-style=ISO,vargreek-shape=TeX]{TeX Gyre Termes Math}

% longtable:

\setcounter{LTchunksize}{10}

% Beginn des Dokuments. %%%%%%%%%%%%%%%%%%%%%%%%%%%%%%%%%%%%%%%%%%%%%%%%%%%%%%%%%%%%%%%%%%%%%%%%%%%%%%%%%%%%%%%%%%%%%%%%%%%%%%%%%%%%%%%%%%%%%%%%%%%%%%%%%%%%%

% Titelblatt. %%%%%%%%%%%%%%%%%%%%%%%%%%%%%%%%%%%%%%%%%%%%%%%%%%%%%%%%%%%%%%%%%%%%%%%%%%%%%%%%%%%%%%%%%%%%%%%%%%%%%%%%%%%%%%%%%%%%%%%%%%%%%%%%%%%%%%%%%%%%%%%

\title{TITEL}
\author{Renke Christian von Seggern}
\date{DATUM}

\begin{document}

\begin{titlepage}

\begin{center}

\vspace*{0.17\paperheight}

{\LARGE Elemente des Gleichungssystems \\ $A \cdot \delta = f$ \par}

\vspace{0.07\paperheight}

{\large Autor:\\ Renke Christian von Seggern \par}

{\large Matrikelnummer: \\ 21027478 \par}

\vspace{0.07\paperheight}

{\normalsize Modul: \\ Projekt Waldökosystemanalyse und Informationsverarbeitung \par}

\vspace{0.02\paperheight}

{\normalsize Abgabedatum: \\ 31.03.2016 \par}

\vspace{0.02\paperheight}

%{\normalsize Dozent: \\ NAME DES DOZENTEN \par}

\end{center}

\end{titlepage}

\newpage

% Inhaltsverzeichnis. %%%%%%%%%%%%%%%%%%%%%%%%%%%%%%%%%%%%%%%%%%%%%%%%%%%%%%%%%%%%%%%%%%%%%%%%%%%%%%%%%%%%%%%%%%%%%%%%%%%%%%%%%%%%%%%%%%%%%%%%%%%%%%%%%%%%%%%

\pagenumbering{Roman}

{
\renewcommand{\MakeUppercase}[1]{#1} % http://tex.stackexchange.com/questions/179966/fancyhdr-chaptermark-and-table-of-contents

%\singlespacing % http://tex.stackexchange.com/questions/56546/how-to-change-spaces-between-items-in-table-of-contents

\tableofcontents

}

% Fließtext. %%%%%%%%%%%%%%%%%%%%%%%%%%%%%%%%%%%%%%%%%%%%%%%%%%%%%%%%%%%%%%%%%%%%%%%%%%%%%%%%%%%%%%%%%%%%%%%%%%%%%%%%%%%%%%%%%%%%%%%%%%%%%%%%%%%%%%%%%%%%%%%%

\newpage

\pagenumbering{arabic}

\section{Konventionen}

\begin{itemize}
\item $i$: Nummer des Knotens ($i = 1,2, \ldots, o$)
\item $m$: Nummer des Iterationsschrittes
\item $n$: Nummer des Zeitschrittes
\end{itemize}

\section{Elemente von $A$}

\begin{math}
  A = 
  \begin{pmatrix}
    b_1 & c_1 & & & & & \\
    a_2 & b_2 & c_2 & & & & \\
    & a_3 & b_3 & c_3 & & & \\
    & & a_4 & b_4 & c_4 & & \\
    & & & \ddots & \ddots & \ddots & \\
    & & & & a_{o-1} & b_{o-1} & c_{o-1} \\
    & & & & & a_o & b_o
  \end{pmatrix}
\end{math}

\subsection{Allgemein für $i \neq 1,o$}

\begin{equation*}
  \begin{split}
    &a_i = - \frac{K_{i-\frac{1}{2}}^{n+1,m}}{\left(\Delta z\right)^2} \\
    &b_i = C_i^{n+1,m} \cdot \frac{1}{\Delta t} + \frac{K_{i+\frac{1}{2}}^{n+1,m}}{\left(\Delta z\right)^2} + \frac{K_{i-\frac{1}{2}}^{n+1,m}}{\left(\Delta z\right)^2} \\
    &c_i = - \frac{K_{i+\frac{1}{2}}^{n+1,m}}{\left(\Delta z\right)^2}
  \end{split}
\end{equation*}

\subsection{Beispielhaft für $i = 2$}

\begin{equation*}
  \begin{split}
    &C_2^{n+1,m} \cdot \frac{\delta_2^m}{\Delta t} - \frac{1}{\left(\Delta z\right)^2} \cdot \left[K_{2+\frac{1}{2}}^{n+1,m} \cdot \left(\delta_3^m - \delta_2^m\right) - K_{2-\frac{1}{2}}^{n+1,m} \cdot \left(\delta_2^m - \delta_1^m\right)\right] \\
    = &C_2^{n+1,m} \cdot \frac{\textcolor{blue}{\delta_2^m}}{\Delta t} - \frac{1}{\left(\Delta z\right)^2} \cdot \left[\left(K_{2\frac{1}{2}}^{n+1,m} \cdot \textcolor{red}{\delta_3^m} - K_{2\frac{1}{2}}^{n+1,m} \cdot \textcolor{blue}{\delta_2^m}\right) - \left(K_{1\frac{1}{2}}^{n+1,m} \cdot \textcolor{blue}{\delta_2^m} - K_{1\frac{1}{2}}^{n+1,m} \cdot \textcolor{green}{\delta_1^m}\right)\right] \\
    = &C_2^{n+1,m} \cdot \frac{\textcolor{blue}{\delta_2^m}}{\Delta t} - \left[\left(\frac{K_{2\frac{1}{2}}^{n+1,m} \cdot \textcolor{red}{\delta_3^m}}{\left(\Delta z\right)^2} - \frac{K_{2\frac{1}{2}}^{n+1,m} \cdot \textcolor{blue}{\delta_2^m}}{\left(\Delta z\right)^2}\right) - \left(\frac{K_{2\frac{1}{2}}^{n+1,m} \cdot \textcolor{blue}{\delta_2^m}}{\left(\Delta z\right)^2} - \frac{K_{1\frac{1}{2}}^{n+1,m} \cdot \textcolor{green}{\delta_1^m}}{\left(\Delta z\right)^2}\right)\right] \\
    = &C_2^{n+1,m} \cdot \frac{\textcolor{blue}{\delta_2^m}}{\Delta t} - \frac{K_{2\frac{1}{2}}^{n+1,m} \cdot \textcolor{red}{\delta_3^m}}{\left(\Delta z\right)^2} + \frac{K_{2\frac{1}{2}}^{n+1,m} \cdot \textcolor{blue}{\delta_2^m}}{\left(\Delta z\right)^2} + \frac{K_{2\frac{1}{2}}^{n+1,m} \cdot \textcolor{blue}{\delta_2^m}}{\left(\Delta z\right)^2} - \frac{K_{1\frac{1}{2}}^{n+1,m} \cdot \textcolor{green}{\delta_1^m}}{\left(\Delta z\right)^2}
  \end{split}
\end{equation*}

\begin{equation*}
  \begin{split}
    &\textcolor{green}{a_2} = - \frac{K_{1\frac{1}{2}}^{n+1,m}}{\left(\Delta z\right)^2} \\
    &\textcolor{blue}{b_2} = C_2^{n+1,m} \cdot \frac{1}{\Delta t} + \frac{K_{2\frac{1}{2}}^{n+1,m}}{\left(\Delta z\right)^2} + \frac{K_{1\frac{1}{2}}^{n+1,m}}{\left(\Delta z\right)^2} \\
    &\textcolor{red}{c_2} = - \frac{K_{2\frac{1}{2}}^{n+1,m}}{\left(\Delta z\right)^2}
  \end{split}
\end{equation*}

\section{Elemente von $\delta$}

\begin{math}
  \delta = 
  \begin{pmatrix}
    \delta_1 \\
    \delta_2 \\
    \delta_3 \\
    \delta_4 \\
    \vdots \\
    \delta_{o-1} \\
    \delta_o
  \end{pmatrix}
\end{math}

\subsection{Allgemein für $i \neq 1,o$}

\begin{equation*}
  \delta_i^m = H_i^{n+1,m+1} - H_i^{n+1,m}
\end{equation*}

\section{Elemente von $f$}

\begin{math}
  f = 
  \begin{pmatrix}
    f_1 \\
    f_2 \\
    f_3 \\
    f_4 \\
    \vdots \\
    f_{o-1} \\
    f_o
  \end{pmatrix}
\end{math}

\subsection{Allgemein für $i \neq 1,o$}

\subsubsection{$h$-basierte Richardsgleichung}

\begin{equation*}
  \begin{split}
    f_i = &\frac{1}{\left(\Delta z\right)^2} \cdot \left[K_{i+\frac{1}{2}}^{n+1,m} \cdot \left(H_{i+1}^{n+1,m} - H_i^{n+1,m}\right) - K_{i-\frac{1}{2}}^{n+1,m} \cdot \left(H_{i}^{n+1,m} - H_{i-1}^{n+1,m}\right)\right] \\
    + &\frac{K_{i+\frac{1}{2}}^{n+1,m} - K_{i-\frac{1}{2}}^{n+1,m}}{\Delta z} - C_i^{n+1,m} \cdot \frac{H_i^{n+1,m} - H_i^n}{\Delta t}
  \end{split}
\end{equation*}

\subsubsection{Gemischte Richardsgleichung}

\begin{equation*}
  \begin{split}
    f_i = &\frac{1}{\left(\Delta z\right)^2} \cdot \left[K_{i+\frac{1}{2}}^{n+1,m} \cdot \left(H_{i+1}^{n+1,m} - H_i^{n+1,m}\right) - K_{i-\frac{1}{2}}^{n+1,m} \cdot \left(H_{i}^{n+1,m} - H_{i-1}^{n+1,m}\right)\right] \\
    + &\frac{K_{i+\frac{1}{2}}^{n+1,m} - K_{i-\frac{1}{2}}^{n+1,m}}{\Delta z} - \frac{\theta_i^{n+1,m} - \theta_i^n}{\Delta t}
  \end{split}
\end{equation*}

\section{Nicht berechenbare Variablen und Elemente}

\begin{itemize}
\item $K_{i-\frac{1}{2}}^{n+1,m}$ für $i = 1$ \\
  $\Rightarrow a_1 = ?$ \\
  $\Rightarrow b_1 = ?$ \\
  $\Rightarrow f_1 = ?$
\item $K_{i+\frac{1}{2}}^{n+1,m}$ für $i = o$ \\
  $\Rightarrow b_o = ?$ \\
  $\Rightarrow c_o = ?$ \\
  $\Rightarrow f_o = ?$
\item $H_{i-1}^{n+1,m}$ für $i = 1$ \\
  $\Rightarrow f_1 = ?$
\item $H_{i+1}^{n+1,m}$ für $i = o$ \\
  $\Rightarrow f_o = ?$
\end{itemize}

% Einfügen des Literaturverzeichnisses. %%%%%%%%%%%%%%%%%%%%%%%%%%%%%%%%%%%%%%%%%%%%%%%%%%%%%%%%%%%%%%%%%%%%%%%%%%%%%%%%%%%%%%%%%%%%%%%%%%%%%%%%%%%%%%%%%%%%%

\newpage

\printbibliography[sorting=nyt]

\end{document}
